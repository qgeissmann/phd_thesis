\begin{figure}[h!]
	\centering   
	\includegraphics[width=0.95\textwidth]{\currfiledir/.\currfilebase.pdf}
	  \caption[Only a few  targeted stimuli causes a homeostatic rebound]{\ctit{Only a few  targeted stimuli causes a homeostatic rebound.}
	\textbf{A}, Number of stimuli delivered over the 12~h of dynamic sleep deprivation. The point colour and x axis represent different intervals (\ie{} time an animal can remain immobile without being startled).
	\textbf{B}, Extra quiescence during the 3~h of rebound, expressed in extra minutes compared to the predicted quiescence (see method subsection~\ref{subsec:mm-rebound}).
	\textbf{C}, Relationship between quiescence during sleep deprivation and number of stimuli delivered in the same period.
	The error bars, in \textbf{A} and \textbf{B}, are 95\% bootstrap re-sampling confidence intervals on the mean (black crosses).
	\textbf{D}, Relationship between lost quiescence during the sleep deprivation and rebound (\ie{} regained quiescence) in the subsequent 3~h.
	The red dashed line at $Y=0$, in \textbf{B} and \textbf{D}, shows the value of rebound expected by chance.
	For the sake of clarity, only five intervals are shown in \textbf{C} and \textbf{D} ($interval \in [20,120,300,540,840]$~s).
	Lines in \textbf{C} and \textbf{D} show linear model fit with standard errors (shaded areas).
	$N_{sex,interval} > 45~\forall~sex \times interval$.
	\label{fig:\currfilebase}
}
\end{figure}
