\begin{figure}[h!]
	\centering   
	\includegraphics[width=0.95\textwidth]{\currfiledir/.\currfilebase.pdf}
	\caption[Prolonged sleep deprivation is effective]{\ctit{Prolonged sleep deprivation is effective.}
		\textbf{A}, Proportion of time engaged in quiescence.
		Females and males are shown on the top and bottom panels, respectively.
		The grey rectangle in the background,
		between t=0.5~d and t=10~d, %%%%%%%%%
		coincides with the permissive time window of stimulus delivery.
		Stimuli were delivered to animals each time they had been immobile for 20 consecutive seconds with the motors of the `optomotor' device
		(see method subsection~\ref{subsec:sd-matmet}). %%%%%%%%
		Controls animals (grey) were undisturbed and spatially interspersed (in the neighbouring tube) with the treated individuals (plum).
		\textbf{B}, Proportion of time engaged in either of the three behavioural states (q: quiescence, m: micro-movement and w:walking) and relative position (from the food, 0, to the cotton wool, 1) during treatment.
		Left and right columns show females and males data, respectively.
		\textbf{C}, Average number of stimulus delivered in each consecutive 30~min, during treatment.
		\textbf{D}, Population averages of the three behavioural states, and position, in the three days following the end of the prolonged sleep deprivation.
		\textbf{E}, Extra quiescence during the 3~h of rebound (blue rectangle in the background of \textbf{A}),
		expressed in extra minutes compared to the predicted quiescence (see method subsection~\ref{subsec:mm-rebound}).
		Values between day 1 and 10 are the average over one day in \textbf{B} and \textbf{C}.
		The shaded areas around the average lines, in \textbf{A-D}, and the error bars in \textbf{E} are 95\% bootstrap resampling confidence intervals on the mean.
		$N_{sex,treatment} > 97~\forall~sex \times treatment$.
		\label{fig:\currfilebase}
	}
\end{figure}
