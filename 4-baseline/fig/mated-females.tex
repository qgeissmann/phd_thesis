\begin{figure}[h!]
	\centering   
	\includegraphics[width=0.95\textwidth]{\currfiledir/.\currfilebase.pdf}
	  \caption[Effect of mating on female behaviour]{\ctit{Effect of mating on female behaviour.}
	\textbf{A}, Proportion of time spent in quiescent (q), micro-moving (m) and walking (w), as well as average position, over the course of the experiment. Females were allowed to interact, for an hour, with a male during the first day (grey bar). Females that had mated and non-mated controls are shown as in green and grey bars, respectively.
	\textbf{B}, Ternary representation of behavioural state values along a circadian day, for  both mated and non-mated populations. In addition, a third, independent, group of non-mated female that had low quiescence was included for comparison purposes.
	\textbf{C}, Hierarchical clustering of the control and mated female groups as well as an independent, low sleeper, third group.
	The average Bhattacharyya distances across all time points were used as a dissimilarity measure.
	The dendrogram was computed using UPGMA (see method subsection~\ref{subsec:mm-clustering}).
	$N_{mated} = 86$, $N_{non\text{-}mated} = 152$ and 	$N_{low\text{-}sleeper} = 110$.
	\label{fig:\currfilebase}
}
\end{figure}
