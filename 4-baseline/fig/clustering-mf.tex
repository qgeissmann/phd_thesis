\begin{figure}[h!]
	\centering   
	\begin{minipage}[t]{0.5\textwidth}
		\vspace{0pt}
		\includegraphics[width=\textwidth]{\currfiledir/.\currfilebase.pdf}
	\end{minipage}\hfill
	\begin{minipage}[t]{0.45\textwidth}
		\vspace{0pt}
	  \caption[Behavioural state modulation]{\ctit{Behavioural state modulation.}
	\textbf{A} and \textbf{B}, Ternary representation of behavioural state values along a circadian day, for female and male populations, respectively.
	Each point represents the average proportion of time spent engaging in quiescence (q), micro-movement (m) and walking (w). 
	The colour of each point and line indicates the time of the day (see the circular colour key). Average were computed for every 15~minutes. Points are circled every hour.
	$N_{male} = 485$ and $N_{female} = 881$.
	\textbf{C}, Hierarchical clustering of a random sample of 400 males and 400 females (a subset was used for performance considerations only).
	The average Bhattacharyya distances across all time points were used as a dissimilarity measure.
	The dendrogram was computed using UPGMA (see method subsection~\ref{subsec:mm-clustering}).
	The three ellipses show partitions of the dendrogram corresponding to males, females, and a heterogeneous outlier group.
	\label{fig:\currfilebase}
	}
	\end{minipage}
\end{figure}
