\begin{figure}[h!]
	\centering   
	\includegraphics[width=0.95\textwidth]{\currfiledir/.\currfilebase.pdf}
	  \caption[The behavr data structure]{\ctit{The behavr data structure.}
	Schematical representation of a \texttt{behavr} object, the central data structure in \texttt{rethomics}.
	In the metadata (left), each row refers to one experimental individual. 
	Columns are metavariables that can be either \emph{required}(\ie{} defined by the acquisition platform) or \emph{user-defined} (\emph{i.e.} arbitrary).
	In the data (right), each row represents a time point, that is the information about one individual at one time-point.
	Data and metadata are internally joined on the \texttt{id} field, the unique identifier of an experimental individual.
	The column names used in this figure are only examples which will differ in practice. 
	This figure is adapted from my own work\cite{geissmann_rethomics_2018}.
%	B: Non exhaustive list of uses of a \texttt{behavr} table (referred as \texttt{dt}). 
%	In addition to operations on data, which are inherited from \texttt{data.table},
%	we provide utilities designed specifically to act on both metadata and data.  
%	Commented examples are prefixed by \texttt{>}.
	\label{fig:\currfilebase}
}
\end{figure}
