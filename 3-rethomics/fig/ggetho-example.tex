\begin{figure}[h!]
	\centering   
	\includegraphics[width=0.95\textwidth]{\currfiledir/.\currfilebase.pdf}
	  \caption[Visualisation with \texttt{ggetho}]{\ctit{Visualisation with \texttt{ggetho}.}
  	Code snippet (left) and resulting plot (right) for an example visualisation of population trend. 
  	The arrows indicate where each code expression is represented on the plot.
  	In this example, the input data, \texttt{dt}, is a behavr table, with variables \texttt{t} (continuous) and \texttt{moving} (logical),
	and metavariables \texttt{treatment} (`Control' or `Treated') and \texttt{sex} (`M' or `F').
    The \texttt{ggetho()} function, the \texttt{stat\_pop\_etho} and \texttt{stat\_ld\_annotations} layers, and the \textbf{scale\_x\_days} scale are part of \texttt{ggetho}, but integrate seamlessly 
    with \texttt{ggplot}, for instance with the faceting mechanism shown here (\ie{} \texttt{facte\_grid}).
	\label{fig:\currfilebase}
}
\end{figure}
