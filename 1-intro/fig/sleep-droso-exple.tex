\begin{figure}[h!]
  \centering   
   \includegraphics[width=0.95\textwidth]{\currfiledir/.\currfilebase.pdf}
  \caption[Example of sleep-time visualisation]{\ctit{Example of sleep-time visualisation.}
	Simplified `sleep trace' representing the probability of observing a sleeping fly, $P(asleep)$, over time.
	This example represents the usual behaviour of a male population.
	Time is often expressed as Zeitgeber Time (ZT) which is $\text{ZT} = Time \pmod{24~h}$, and zero corresponds to the start of the day (phase L).
	In this example, the light regime is 12~h of light and 12~h of darkness (12:12~h, L:D).
	Males tend to have a bimodal activity  with a morning and an evening peak, and are inactive in between.
  \label{fig:\currfilebase}
  }
\end{figure}





