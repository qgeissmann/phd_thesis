\begin{figure}[h!]
  \centering   
   \includegraphics[width=0.95\textwidth]{\currfiledir/.\currfilebase.pdf}
  \caption[The two-processes model of sleep regulation]{\ctit{The two-processes model of sleep regulation.}
	Theoritical example illustrating the model for a diurnal animal, over three days, in two scenarios:
	either it is allowed to sleep throughout (\textbf{A}), or it is prevented to do so during the first night (\ie{} sleep deprivation, red background, \textbf{B}).
	The propensity to sleep is the sum of two processes: $C$ and $S$. $C$, the clock, is a periodic function that does not depend on sleep pressure.
	$S$, represents the homeostat. It decreases when sleep occurs (blue), and increases when it does not.
	Sleep occurs when $C+S > H^+$, and an animal wakes up when $C+S < H^-$.
  	\textbf{A}, In normal circumstances, $C$ and $S$ add up to a periodic function that is the sleep propensity, and sleep is the realisation of such propensity.
  	\textbf{B}, During sleep deprivation, the clock ($C$) is unaltered, but the sleep homeostat ($C$) increases asymptotically. 
	When sleep deprivation is released, the animal may still have an overall sleep pressure ($C+S$) greater than ($H^+$) and will dissipate $S$ exponentially.
	In this example, there is such a sleep rebound, during the day, immediately after sleep deprivation.
  \label{fig:\currfilebase}
  }
\end{figure}





