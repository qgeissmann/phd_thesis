\edef\figdir{\currfiledir/fig}

\chapter{Introduction} \label{intro}
	
\epigraph{
	`For six days and seven nights, come, do without
slumber.'}{--- Uta-napishti, in \emph{The Epic Of Gilgamesh}\cite[tablet~XI~209]{george_epic_2002}}
\section{General introduction}
\subsection{Sleep, the history of a mystery}
\subsubsection{In the oldest tale we know of}
From 668~BC to 627~BC, the Neo-Assyrian empire was ruled by King Ashurbanipal.
Proud of his literacy and aware of the power of knowledge, he ordered his armies to collect all clay tablets they could find
throughout Mesopotamia and he stored them in his capital, Nineveh. 
After his death, civil wars weakened the state and, in 612 BC, Babylonian invaders eventually burned the city to the ground.
Ashurbanipal's library collapsed, and the ancient tablets were buried and forgotten in its ruins\cite{menant_bibliotheque_1880,polastron_books_2007}.


Until, in the mid-nineteenth century, Austen Henry Layard discovered the remains of Nineveh\cite{layard_discoveries_1853}.
A few years after starting the excavations, his assistant, Hormuzd Rassam, came across a very special set of tablets.
They told the story of Gilgamesh, the semi-fictional king of the great city of Uruk\cite{menant_bibliotheque_1880}.
This tale, the `Epic of Gilgamesh', is considered the most ancient piece of literature known to date\cite{tigay_evolution_2002}.

In many respects, the epic reflects how the world was seen and understood by the people who forged the first civilisations.
It is quite clear that sleep was already seen as a fascinating, but also mysterious, topic. 
In particular, several references show that sleeplessness was thought of as a supernatural trait, comparable to immortality.
For instance, Gilgamesh slays Humbaba, the ferocious giant that never sleeps\cite[I~239]{george_epic_2002}.
The epic climaxes when Gilgamesh, who seeks eternal life, is challenged to defeat sleep\emd{}the younger brother of death\emd{}to prove himself.
However, despite his great strength, he fails to remain awake even for a single night\cite[XI~209]{george_epic_2002}. 
Altogether, the tale hints that \emph{sleep is an intrinsic limitation which is inherent to the human\emd{}and maybe to the mortal\emd{}condition}.

\subsubsection{Sleep and death, a cultural perspective}

There are many examples showing a deep interest in sleep in other mythologies and cultures,
and the interested reader will find ample material published on the matter\cite{hohne_mythology_2009,monge_sommeil_2005}. 
In connection with the present thesis, I was particularly interested in one aspect: the analogy, already found in the Epic of Gilgamesh, between the two mysteries that are sleep and death, and wanted to illustrate their\emd{}sometimes implicit\emd{}association with a few examples drawn from culture and mythology.

In ancient Egypt, death was seen as an `eternal sleep' and, during sleep, the living could communicate with the world of the dead\cite{asaad_sleep_2015}.
In Greek mythology\emd{}for instance, in Hesiod's theogony\emd{}Hypnos, the god of sleep, is the half-brother of Thanatos, the personification of death\cite{hesiod_theogony_2006}.
Homer and Virgil themselves use the euphemism of `deep sleep' to name death\cite{ragon_espace_2012}.
Later, in the dedication to his \emph{Natural History}, Pliny the Elder writes that `life properly consists in being awake'\cite[book~I]{pliny_the_elder_natural_1855}.
In fact, for him, sleeping could, in some cases, rejuvenate.
For instance, he surmises that `[the] old age of [dormice] is put an end to by their winter's rest, when they conceal themselves and sleep'\cite[book~VIII, chap.~82]{pliny_the_elder_natural_1855}.

The metaphor of sleep is also part of the Judeo-Christian tradition. 
For instance, for Paul the Apostle, graveyards are large dormitories\cite{ragon_espace_2012}.
Later, Milton will also question whether sleep and death are two degrees of the same phenomenon: `the lifeless body does not sleep, unless inanimate matter can be said to sleep'\cite[p.~285]{milton_treatise_1825},
In \emph{Sleep And Societies}, Simon Williams points out that the intimate link between sleep and death is present throughout literature, for instance in the work of Shelley, Donne, Bunyon and Shakespeare\cite[p.~150]{williams_sleep_2013}.

%One may wonder why this spiritual and philosophical association between the two fundamental mysteries that are sleep and death seems so
%ancient and prevalent. 
%Even though sleep, unlike death, is reversible, it is no surprising that the two phenomena have been so closely associated, death being seen as a deeper form of sleep.
%Both states involve a lack of movement and responsiveness, and maybe more fundamentally are inescapable.

For the modern reader\emd{}who, I am sure, knows more about physiology than the ancients did\emd{}it may be intuitive to comprehend sleep and death as two different, even dichotomous, processes.
In particular, sleep is reversible, whilst death is not.
However, in the old times, it seems as though the similarities, rather than the differences, between these two mysteries were highlighted.
This may arise partly from the fact that both states involve closed eyes as well as a lack of movement and responsiveness.
Perhaps an even more fundamental resemblance, though, is that both sleep and death appear ineluctable.

\subsection{Phenomenological definition of sleep}	
\subsubsection{Towards an open definition}

Often, as I finish a presentation of my work on sleep in fruit flies, an inquisitive listener from the audience
takes the opportunity to raise their hand and challenge its very premise: my \emph{definition of sleep},
and whether I should use the word `sleep' at all.
Sometimes, I am even charged guilty with the sin of anthropomorphism, 
as if using the same name for two processes implied a complete identity of functions, origin and implementation.

I find comfort in the realisation that, in behavioural biology, others have faced similar criticisms.
A famous example is the use of the word `culture' by primatologists, a term historically reserved for humans, 
but now accepted\emd{}at least, amongst scientists\emd{}as a phenomenon with wider scope\cite{laland_question_2009,whiten_scope_2011}.

In his prologue to \emph{The Ape And The Sushi Master}, Frans de Waal  delivers a powerful argument in favour of `open definitions' in general\cite{waal_ape_2001}.
He argues that we tend to overestimate the risk of anthropomorphism, but dismiss the benefits of open definitions.
He anecdotally points out that we use the term `bipedal' to describe both humans and chickens, whilst their implementation is rather different.
I would add, to pick an example closer to my field of research, that we speak of insect `respiration', whilst they have neither lungs nor red blood cells\cite{slama_new_1988,nation_insect_2008}, therefore without implying macroscopic identity with the physiological processes that happen in, say, mammals.
In fact, calling two processes or structures the same does not even necessarily involve a shared evolutionary origin.
For instance `wings' and `eyes' have evolved multiple times\cite{land_evolution_1992,stern_genetic_2013}, but sharing terminology is important to fully understand the convergence of innovations to similar evolutionary and physical constraints.

Our comprehension of numerous biological processes has resulted from a translation between model organisms.
Arguably, using different names reduces visibility across fields and ultimately may prevent collaboration.
For instance, if one had restricted the notion of immunity to its adaptive arm, or even to the presence of antibodies\emd{}which are exclusive of vertebrates\cite{flajnik_origin_2010}\emd{}, it could have been difficult to initiate, and eventually translate, the seminal work that was done on the innate immunity of \dmel{}\cite{hoffmann_innate_1995}.

The above processes (\ie{} immunity and respiration) and
structures (\ie{} bipedal legs, wing and eyes), however, differ fundamentally from sleep 
insofar as \emph{their function is known}, or at least postulated,
whilst the role of sleep is, at best, elusive\cite{siegel_sleep_2009,joiner_unraveling_2016}.
In other words, we cannot provide an open definition of sleep that would  be based upon 
its main function, because we simply do not know it.
In fact, we often define sleep as a lack of function, for instance a loss of awareness to the environment, a lack of movement and so on (see subsection~\ref{sec:sleep-def} and \cite{campbell_animal_1984}). 
It therefore seems as though, as long as we have no compelling evidence for a shared role for sleep across all species, we should be very cautious about using this term.

There are, however, widespread phenomena that have been defined and extensively studied, without their function being known \emph{a priori}.
For instance, senescence, and eventually death, are broadly accepted as ubiquitous and necessary.
The function of ageing\emd{}and, in fact, whether it has one at all\emd{}has nevertheless long been debated by evolutionary biologists\cite{partridge_optimally_1993,hughes_evolutionary_2004,rose_evolution_2008}.
Instead of postulating a function, and making the risky assumption of a phylogenetic conservation, we use a \emph{`phenomenological' definition}.
That is, we study ageing as an observable phenomenon whereby the functions of an organism gradually deteriorates until it eventually dies\emd{}even
though it manifests itself in various forms in different organisms and has multiple levels (\ie{} molecular, cellular and physiological)\cite{johnson_molecular_1999,sun_mitochondrial_2016}.

The field of ageing, despite a fundamentally function-agnostic framework, 
has historically drawn from a comparative approach,
both describing similarities and differences between organisms\cite{hughes_evolutionary_2004,austad_comparative_2009}.
This example shows that phenomenological definitions, too, can be the substrate for a deeper understanding of biological processes.
Therefore, if we would like to study sleep with the lens of comparative biology, and without presupposing a function, we ought to provide a phenomenological definition.

\label{sec:sleep-def}
\subsubsection{From behaviour to brainwaves}
Historically, sleep was first scientifically investigated in mammals, including humans.
It was then defined as an overt \emph{behaviour}\cite{pieron1913probleme}.
However, the observation of macroscopic behavioural states was to be superseded by the rise of a new science: electrophysiology.

Indeed, in the late 1930s, Frédéric Bremer noticed that sleep behaviour correlated with changes in the electrical activity of the brain
\cite{bremer1937activite,kerkhofs_frederic_2000}.
From then onwards, and with the advance of the \gls{eeg}, the field progressively\emd{}and eventually almost entirely\emd{}adopted a quantification of sleep based on brainwaves.
\gls{eeg} represented a non-invasive, but also relatively high-throughput and objective alternative to behavioural observation.
In addition, it could be used for, and compared between, a variety of model organisms and even reveal cryptic states, such as different sleep stages\cite{moruzzi_brain_1949,swett_effects_1968,borbely_sleep_1981,sallanon_long-lasting_1989}.
Ultimately, \gls{eeg} became the \emph{de facto} standard to study sleep in vertebrates,
to the point that brainwaves became implicitly included in the definition of sleep.


\subsubsection{From brainwaves to behaviour}

In the last few decades, a growing interest in a comparative approach\cite{campbell_animal_1984,siegel_all_2008} and in `simpler'\emd{}non-avian and non-mammalian\emd{}models of 
sleep\cite{hendricks_need_2000} arose, which prompted the need for an inclusive definition of sleep. 

Although \gls{eeg} has become the standard for studying sleep in mammals and bird, it cannot be used to study sleep in other organisms such as insects or even fish, as brainwaves originate from structures that are simply not present in most animals\emd{}namely, cortical regions.

Since brainwaves are merely a convenient, but specific, read-out of a process that is mostly unapparent, it would have been too restrictive to limit the definition of sleep to the exclusive observation of \gls{eeg}.
To use an intentionally provocative analogy, it would have been comparable to stating that sleep necessarily implies eye closure, \emph{de facto} excluding most animals since eyelids\emd{}and, for that matter, eyes\emd{}are rather rare in animals.

It, therefore, became necessary to open the scope of the definition of sleep to one
that could, in principle, be applicable to a wide range of life forms.
This concern led to a return to a \emph{behavioural} definition of sleep, which is based on three fundamental observations\cite{pieron1913probleme,campbell_animal_1984,siegel_sleep_2009}.

Firstly, it implies a relative \emph{immobility} (\ie{} quiescence) with a particular posture\cite{flanigan_sleep_1973} that can be species-specific.
For instance, apes, including humans, lie down and close their eyes\cite{campbell_animal_1984}.
Importantly, such quiescence must be rapidly reversible, which differentiates sleep from dormant states (\eg{} torpor, hibernation and æstivation), but also from pathological states such as coma\cite{siegel_sleep_2009}.
In addition to the posture, sleep may involve a preferred resting site\cite{durie1981sleep}.

Secondly, during sleep, organisms must have a \emph{reduced awareness} of their surrounding environment.
Empirically, this translates in a higher arousal threshold\emd{}\eg{} a stimulus that would cause an awake animal to escape will not startle a sleeping one
\cite{huber_sleep_2004}.

Thirdly, sleep is defined as a \emph{homeostatic} process. 
In other words, the loss of sleep will increase the need for sleep, and may eventually lead to a so-called `rebound', during which sleep propensity and depth are increased\cite{allada_molecular_2017}.
Importantly, the recovery of sleep loss is not expected to be\emd{}and rarely is\emd{}as long\cite{siegel_sleep_2009} as the total sleep deficit. 

Another interesting consideration is that sleep is often intimately linked to an animal's circadian clock, and is almost always studied in this context.
The propensity of an individual to sleep is then thought of as the addition of two processes (see subsection~\ref{sec:two-process} and \cite{borbely_two_1982}): the homeostatic drive (process S) and the clock (process C). 
In other words, sleep is not modulated by its homeostat only, but also, and to a large extent, by the internal clock of an animal.
Sleep is therefore generally recurrent (\ie{} daily), and happens throughout life\cite{vorster_sleep_2015}.

\subsection{Ubiquity of sleep}
The question of the ubiquity of sleep amongst animals is very ancient\emd{}and, I would argue, entangled with its definition.
For instance, in his monumental `Natural History', Pliny the Elder writes a short chapter entitled `the sleep of animals', where he asserts that all animals `with eyelids', but also fish and insects undoubtedly sleep\cite[book~X, chap.~97]{pliny_the_elder_natural_1855}, a view that can actually be traced back to Aristotle's
`History of Animals'\cite[\eg{} book~IV, chap.~10]{aristotle_history_1965}.

Using the permissive behavioural definition I have just presented,
it becomes conceivable to investigate sleep in a vast array of organisms, throughout phylogeny.
Various authors, who have the reviewed literature on the subject, consider sleep a ubiquitous phenomenon\cite{campbell_animal_1984,cirelli_is_2008,joiner_unraveling_2016}.
It is also even presented as  an evolutionary `conserved' phenomenon\cite{vorster_sleep_2015,zimmerman_conservation_2008}\emd{}\ie{}
the common ancestors of all Metazoans would have already slept.
Such views are, however, equivocal as others maintain that the evidence for ubiquity is not satisfying and that behavioural sleep could simply not be characterised in some models\cite{siegel_all_2008}.
In this section, I will attempt to provide an overview of the evidence\emd{}or lack thereof\emd{} for the existence of sleep in several taxa from both perspectives.

My purpose is not to make a long bestiary of all species in which sleep has been described\emd{}as this has already been done multiple times, for instance in \cite{campbell_animal_1984,lesku_phylogenetic_2006,zimmerman_conservation_2008,vorster_sleep_2015,joiner_unraveling_2016}\emd{}, 
but rather, for the purpose of this thesis, to summarise the main properties of sleep by taxum.
As a comparative biologist, I will focus on the description of sleep in distant and atypical\emd{}and sometimes controversial\emd{}models such as insects, molluscs, roundworms and jellyfish,
but first, will very briefly summarise the\emd{}less contentious\emd{}work on vertebrates, starting with mammals.


\subsubsection{Mammals}

As I explained above, sleep in mammals has mostly been, and still is, investigated through the use of \gls{eeg}.
The study of brainwaves has led to the separation of sleep in two main states: \gls{rem} and \gls{sws}.
The former, also known as paradoxical sleep, is characterised by high-frequency brainwaves that are indistinguishable from wake state, and accompanied with a loss of muscle tone (except for the eye muscles which, in humans at least, are active),
whilst the latter involves brain activity with macroscopically lower frequency peaks.
Most of the experimental evidence for sleep in mammals comes from human and several domesticated land species\cite{lesku_phylogenetic_2006,siegel_all_2008}, namely rat, mouse, cat, dog, and several species of monkeys\emd{}though an increasing number of studies have investigated sleep in wild animals\cite{staunton_mammalian_2005}.
Interestingly, marine mammals have long puzzled sleep researchers\emd{}so much so that I thought they deserved their dedicated subsection.

\paragraph*{Land mammals}

The fact terrestrial  mammals sleep is widely accepted (reviewed, for instance, in \cite{campbell_animal_1984, elgar_sleep_1988,tobler_is_1995,nicolau_why_2000,staunton_mammalian_2005}). 
Indeed, multiple experiments have shown that they exhibit all the features of the definition of sleep.
For instance, sleep has extensively been studied in rodents, in particular in rats and mice that have long periods of inactivity during the day (since they are nocturnal) that correlate with electrophysiological measurements.
When engaged in such states, they have been shown to be less responsive to external stimuli such as vibrations\cite{davis_study_1972} or sounds\cite{neckelmann_sleep_1993}\emd{}though arousal threshold differs between models\cite{van_twyver_sleep_1969}.
When sleep-deprived, they recover the lost sleep, with more, but also deeper, quiescence\cite{mouret_paradoxical_1969,everson_sleep_1989}.

There is also a vast literature specifically in human, the description of which goes beyond the scope of this general introduction.
For completeness purposes, I will, however,\emd{}very superficially, I am afraid\emd{}mention the singularity of the work on human sleep.
Firstly, several large-scale epidemiological studies have been performed in order to statistically link sleep to other variables such as genetics\cite{partinen_genetic_1983,watson_genetic_2006}, and age, sex, lifespan and health conditions\cite{ohayon_meta-analysis_2004, kurina_sleep_2013,akerstedt_sleep_2017,akerstedt_sleep_2018}.
Then, psychology experiments that address the relationship between sleep and high-order cognitive tasks\cite{thomas_neural_2000} or emotions\cite{yoo_human_2007} have been carried out.
Furthermore, idiosyncratic phenomena such as dreaming\cite{siclari_neural_2017} or sleepwaking\cite{bakwin_sleep-walking_1970,petit_childhood_2015} have been under scrutiny.
Lastly, there are interesting considerations on how sleep interacts with culture,
for instance with the use of artificial light\cite{begemann_daylight_1997}, and society\cite{yetish_natural_2015,de_la_iglesia_access_2015}.

In addition to the corpus of studies, with clinical applications, on these conventional, domesticated, models, there are also many research articles on sleep in other mammals (reviewed in \cite{campbell_animal_1984,staunton_mammalian_2005}).
They show a large variability in the amount of sleep.
For instance elephants\cite{tobler_behavioral_1992} and giraffes\cite{tobler_behavioural_1996} reportedly sleep as little as 4~h, whilst 
species of bats would do so almost 20~h each day\cite{zepelin_mammalian_1974,campbell_animal_1984}.
Although most authors have worked with in animals kept in zoos\cite{siegel_all_2008}, an increasing number of studies have managed to record sleep in the field\cite{aulsebrook_sleep_2016}.

\paragraph*{Marine mammals}
Between 50 and 30 million years ago, three groups of 
mammals\emd{}namely, the Cetaceans, the Pinnipedes (\eg{} seals and walruses) and the Sirenians (\ie{} dugongs and manatees)\emd{}
independently diverged from their respective land ancestors and evolved specific adaptations to the marine environment\cite{berta_marine_2005}.
Since all marine mammals need to regularly emerge at the surface to breathe, 
sleep researchers have long wondered how this seemingly active behaviour was compatible with sleep\cite{lilly_animals_1964}.

Cetacean sleep is particularly well studied (reviewed in \cite{lyamin_cetacean_2008}) and has several unique features.
Firstly, the presence of \gls{rem} sleep has not been shown conclusively\cite{mukhametov_black_1997,lyamin_cetacean_2008}.
Secondly, cetaceans seem able to sleep whilst moving\cite{sekiguchi_sleep_2006}.
Thirdly, they  are capable of `unihemispheric' \gls{sws} sleep, a state where only one hemisphere of the brain shows features of sleep, the other being electrophysiologically awake\cite{lyamin_relationship_2004}.


Noticeably, some mother cetaceans have been reported to be constantly active after giving  birth\cite{lyamin_animal_2005}, which indicates that sleep can evolve to be optional, at least under some conditions.

\subsubsection{Birds}

Birds (a taxum with approximately 10,000 species) are very interesting for sleep research in, at least, three ways.
Firstly, there is a long lasting tradition of studying them in the field, which constitutes an unusual opportunity to investigate sleep in ecologically relevant contexts.
Secondly, some migratory birds are known to fly over long distances without ever landing, which has puzzled scientists.
Lastly, birds have evolved, independently of mammals, cortical structures, which makes \gls{eeg}
applicable to them, and explains the vast literature on avian sleep\cite{rattenborg_avian_2009}.

One seminal study on the barbary dove (\emph{Streptopelia risoria}) showed that exposure to a predator created a sleep debt that was then
compensated\cite{lendrem_sleeping_1984}. 
Multiple studies following this early work have led to the broad consensus that,
like mammals, birds have both \gls{rem} and \gls{sws}\cite{rattenborg_avian_2009}.

Like sea mammals, several species of bird, exhibit unihemispheric \gls{sws}, including blackbirds (\emph{Turdus merula})\cite{szymczak_study_1996}, mallards (\emph{Anas platyrhynchos})\cite{rattenborg_half-awake_1999} and others (see table 1 in\cite{rattenborg_avian_2009})
In addition, there is some evidence that some gliding species can sleep in flight\cite{rattenborg_evidence_2016}.

A notable study showed that, during the breeding season, some male pectoral sandpipers (\emph{Calidris melanotos}) mate
with multiple females with very little rest, and that this sleep loss is neither compensated nor  does it impact fitness\cite{lesku_adaptive_2012}, fuelling controversy on the idea that sleep is a fundamental necessity.

\subsubsection{Fish}

\paragraph*{Early work}

There are approximately $25,000$ species of Teleost (bony) fish with very diversified ecological niches\cite{hoegg_phylogenetic_2004}.
An early study on a small number of goldfish (\emph{Carassius auratus}) and perches (\emph{Cichlosoma nigrofasciatum}), 
has shown that both species experienced rest, and that light exposure during the night could increase their activity,
with a subsequent rebound\cite{tobler_effect_1985}.
However, no changes in responsiveness were characterised.
Behavioural states, including quiescence, were determined by human observation.

\paragraph*{Zebra fish}

The zebrafish (\emph{Danio rerio}) is an important and genetically tractable model that has traditionally been used in developmental biology,
genetics, toxicology\cite{hill_zebrafish_2005, lieschke_animal_2007} and, more recently, behavioural genetics\cite{norton_adult_2010}.

There is a growing pool of accepted evidence that zebrafish exhibit a sleep behaviour\cite{zhdanova_melatonin_2001, zhdanova_sleep_2006, yokogawa_characterization_2007, zhdanova_sleep_2011}.
Some of the studies that investigated sleep in zebrafish also show surprising neurochemical similarities between mammalian and fish sleep. 
For instance, some hypnotic drugs affect sleep in both models\cite{zhdanova_melatonin_2001,zhdanova_sleep_2011}.

It is important to notice that most research is done on very young fish larvae (typically, 7--14~days old) as opposed to adult animals.
Since zebrafish larvae are very small and aquatic, electrophysiological measurements are not performed.
Instead, sleep is generally assessed by high-throughput automatic behavioural scoring of video recordings.

\paragraph*{Blind cavefish}
More recently, a very interesting model of sleep was suggested: \emph{Astyanax mexicanus}, a species
in which several populations have independently adapted to life in the absence of light, and eventually lost their eyes\cite{duboue_evolutionary_2011,jaggard_hypocretin_2018}.
This model is particularity interesting as the circadian clock of the cave-dwelling populations is altered\cite{beale_circadian_2013}, prompting questions of the evolution of sleep when biological rhythms are dampened.
The blind populations were shown to have rapidly evolved towards very low amounts of sleep\cite{duboue_evolutionary_2011}. The same team of researchers further characterised a neuronal pathway that is involved in this reduction of sleep\cite{jaggard_hypocretin_2018}. 
This comparative approach shows that the amount of sleep can be a labile and rapidly evolving phenotype.

\subsubsection{Arthropods}

Arthropods are the most diversified phylum with an estimated five million species, including, amongst others, all insects and crustaceans\cite{odegaard_how_2000}. 
The diversity of their ecological niches as well as their overall biomass and environmental impact\cite{stork_abundance_1993,ganihar_biomass_1997} makes them an unavoidable group\emd{}at least in the context of comparative biology\emd, but sleep has only been studied in a few of them.

%The interest for the study of sleep in arthropods seem to date as far as the the late 1960s \cite{andersen1968sleep}\footnote{Though I did not manage to check this reference myself}.
\paragraph*{Early work on cockroaches and scorpions}
A series of landmark experiments on several species of cockroaches showed that, at least under some conditions, immobile body posture correlated with increased arousal threshold\cite{tobler_effect_1983,tobler_24-h_1992}.
Furthermore, animals that were forced to remain active exhibited a subsequent inactivity rebound.
The same authors were able to replicate their findings in scorpions\cite{tobler_rest_1988}.
More recently, an interesting study concluded that sleep deprivation was deleterious to another species of cockroach, even suggesting a lethal effect\cite{stephenson_prolonged_2007}.
In the early studies, behavioural states, including quiescence, were determined by human observation. 

\paragraph*{Evidence in crayfish}
The crayfish is a historical model in neurobiology, in particular,
due to the extensive study of the encoding of its fast escape behaviour\cite{edwards_fifty_1999}.
An interesting study, which investigated sleep in \emph{Procambarus clarkii}, found that  quiescence behaviour was consistent with an elevated response to mechanical stimuli and that keeping individuals awake (using bubbling water) resulted in a rebound\cite{ramon_slow_2004}.

Furthermore, electrophysiological recordings of resting crayfish indicated a consistent slow rhythm, 
which was later studied by others\cite{mendoza-angeles_slow_2007,mendoza-angeles_slow_2010} and suggested as an analogue of mammalian \gls{sws}.



\paragraph*{Honey bees}
The honey bee, \emph{Apis mellifera}, has a remarkable
brain plasticity and is notoriously capable of unique cognitive abilities such as various forms of learning, advanced navigation and communication\cite{srinivasan_honey_2010,menzel_honeybee_2012,howard_numerical_2018}.
In addition, circadian clocks are essential for bees as they use it, in conjunction with the position of the sun, to navigate\cite{frisch_dance_1967,homberg_search_2004}. 
For these reasons, they constitute a compelling model for the study of sleep and circadian rhythm in the context of learning and plasticity.


It was shown that bees have long bouts of inactivity that correlate with an increased arousal threshold\cite{kaiser_neuronal_1983,kaiser1985comparative,kaiser1988electromyographic}.
Furthermore, it was suggested that sleep deprivation led to a subsequent alteration of their behaviour, with bees showing deeper\emd{}but not longer\emd{}sleep episodes\cite{sauer_sleep_2004}.

Nurse bees and foragers are reported to have different sleep patterns\cite{eban-rothschild_differences_2008}.
Foragers are also capable of dynamically allocating sleep time according to food availability\cite{klein_work_2011}.

Sleep deprivation is thought to impact cognitive functions. 
Namely, impairing the waggle dance\cite{klein_sleep_2010} and the
consolidation of odour memory\cite{hussaini_sleep_2009}.
In addition, the presentation of a context odour during sleep has been shown to enhance memory consolidation of bees\cite{zwaka_context_2015}\emd{}a phenomenon previously observed in mammals\cite{rasch_odor_2007}.


\paragraph*{\droso}
The fruit fly \dmel{} is one of the most widely used biological models.
Historically, it has primarily  been  instrumental to the field of genetics and developmental biology, but grew as a model for, amongst a long list,
ageing, population genetics, immunity and behaviour\cite{hoffmann_innate_1995}.

Indeed, from the late 1970s, Seymour Benzer used it as a tool to demonstrate that behaviours could be determined by genes\cite{tully_discovery_1996}.
Perhaps the most successful achievement of this approach was the uncovering of the molecular mechanisms of the circadian clock.

It is, however, only later, in the early 2000s, that \dmel{} emerged as a model of sleep, with two seminal articles showing that its quiescence fulfilled the three criteria of the behavioural definition of sleep\cite{hendricks_rest_2000,shaw_correlates_2000}. 
In order not to interrupt our journey, from branch to branch, into the phylogeny of sleep, 
I will restrain my urge to immediately elaborate on the literature that followed these seminal studies and refer the impatient reader to section~\ref{sec:sleep-droso}.



\subsubsection{Nematodes}

In the last decade, the existence of a sleep-like state has also been shown the roundworm, \emph{Caenorhabditis elegans}\cite{trojanowski_call_2016}.
This widely used biological model only possesses a simple nervous system with barely more than 300 neurons\cite{white_structure_1986},
which makes it a compelling candidate to study the genetics and neuronal
origins of behaviours\cite{bargmann_beyond_2012}.

\emph{C. elegans} develops by going through discrete moulds, and it had been noticed that, during the few hours that precede each ecdysis (\ie{} the process of moulting), worms exhibit a period of quiescence that was named `lethargus'\cite{cassada_dauerlarva_1975}. 
Later, it was proposed that lethargus is a `sleep-like state'\cite{raizen_lethargus_2008}.
Indeed, the latency of a lethargic worm to respond to a relevant chemical stimulus is increased.
In addition, lethargus, seems homeostatically regulated since mechanically or chemically stimulating quiescent animals results in a subsequent decrease of activity\cite{raizen_lethargus_2008,iwanir_microarchitecture_2013}.
Lethargus is therefore seen as a `developmentally-timed' form of sleep\cite{trojanowski_call_2016}.

Another sleep-like state was identified in worms which, in contrast, is induced by stress.
This `stress-induced' sleep occurs after the exposition to a stressor and also involves, at least, an elevated response threshold\cite{hill_cellular_2014}.

Crucially, both types of sleep described in worms are \emph{not} regulated in a circadian manner.
Stress-induced sleep is fundamentally different from sleep described in other animals insofar as it is not constitutive, but induced. Developmentally-timed sleep is also singular in that it is only described in the immature larvae rather than in the adult animal.

%method

\subsubsection{Molluscs}
Molluscs are a major phylum with nearly $100,000$ species, most of them aquatic ones, but also some adapted to life on land (\eg{} snails)\cite{rosenberg_new_2014}.
From a neurobiological perspective, they are particularly interesting as they feature a range of cognitive capabilities, from sessile organisms, such as Bivalvia, with a limited nervous system, to Cephalopoda and their astonishing learning capabilities.

\paragraph*{Aplysia}

In the 1960s, it was using a mollusc, the sea slug \emph{Aplysia californica},
that Eric Kandel and his team were able to reveal the cellular basis of simple memories and habituation\cite{kandel_mechanism_1964,milner_cognitive_1998}.
Recently, a study investigated sleep in this landmark model, showing that \emph{Aplysia}
exhibits prolonged periods of quiescence, which are associated with place preference\cite{vorster_characterization_2014}.
In addition, its latency to respond to ecologically relevant stimuli, such as the addition of food or an increase in salt concentration, was increased when quiescent. 
Finally, a 12~h sleep deprivation in the night, led to a large rebound during the following day.
Rest state was determined by human observation.

\paragraph*{Lymnea}

An interesting study was carried on the great pond snail, \emph{Lymnaea stagnalis}\cite{stephenson_behavioural_2011}. The authors videotaped the behaviour of animals for almost three months and described the occurrence of quiescence bouts lasting several minutes.
They also reported that behavioural rest correlated with higher response thresholds, to both appetitive and aversive stimuli.
Crucially, they failed to characterise an underlying homeostatic control of sleep\emd{}\ie{} no rebound. 
In addition, the organisation of quiescence 
bouts was characteristic of a random walk rather than regulated in a circadian fashion.
In this study, quiescence was determined by human observation.

\paragraph*{Cephalopoda}


Cephalopods are noticeable amongst molluscs for their outstanding learning ability. 
A study reported the existence of period quiescence in \emph{Octopus vulgaris} that correlates with a decreased brain activity\cite{brown_brain_2006}.

Another article, this time in the cuttlefish, \emph{Sepia officinalis}, characterised a resting behaviour that is, to a limited extent, homeostatically regulated.
Indeed, compensatory quiescence was observed after a 48~h sleep deprivation\emd{} which was performed using a monitor to display constant visual stimuli\cite{frank_preliminary_2012}.
%pseudoreplication!

Neither studies reported a change in arousal threshold during quiescence and both scored behaviour manually, using explicit criteria.

\subsubsection{Cnidaria}

A recent study has pushed further the boundaries of the field with the discovery of a `sleep-like' state in the jellyfish \emph{Cassiopea spp}\cite{nath_jellyfish_2017}.
The authors found, firstly, that the pulsatile activity of \emph{Cassiopea} was much lower, and had long pauses, at night.
Secondly, during their quiescence bouts, animals showed lower responsiveness to a mechanical stimulus.
Lastly, recurrent perturbation during the night (\ie{} a jet of water every 20~min), which increased activity, resulted in a subsequent rebound after the perturbation had stopped.


These findings is particularly interesting as the Cnidaria (jellyfishes, but also sea anemones and corals) branched away from the Bilateria (\eg{} arthropods, molluscs and vertebrates) 600 million years ago, before the latter evolved a central nervous system, and therefore have only a simple, distributed, network of neurons\cite{dunn_animal_2014,katsuki_jellyfish_2013,arendt_nerve_2016}, hence challenging the notion that sleep evolved as a requirement for the central nervous system.
%method

\subsection{Functions of sleep}
\label{sec:function}
Since, as I just described, sleep appears so widespread throughout phylogeny, it is plausible that is could serve a core set of functions that all animals would need.
There has been a broad range of suggestions as to what such roles could be:
development of the brain, learning and memory, metabolism, immunity and others.
However, there is no universal consensus on one single function of sleep\cite{cirelli_is_2008, siegel_sleep_2009}.
Before presenting the three mains hypotheses in the literature on this subject, I would like to address, 
perhaps, the most obvious question at this stage: \emph{is sleep a vital need}?

\subsubsection{Is sleep a vital need?}

There is no doubt that the lack of sleep has widespread physiological consequences in some organisms.
However, the idea that sleep is vital\emd{}to the extent that sleep deprivation eventually causes death\emd{}is not consensual.
For example, Chiara Cirelli and Giulio Tononi have argued that sleep is essential\cite{cirelli_is_2008},
whilst Jerome Siegel pointed out the lack of evidence to support such a view\cite{siegel_sleep_2009}.
Indeed, the literature on the prolonged effects of sleep restriction is far from comprehensive\emd{}considering the central importance of the question\emd{}and is partly dated.
Experiments addressing this matter have hitherto been reported only in a handful of model organisms:
dogs\cite{bentivoglio_pioneering_1997}, 
rats\cite{rechtschaffen_physiological_1983, rechtschaffen_sleep_1995}, 
pigeons\cite{newman_sleep_2008},
cockroaches\cite{stephenson_prolonged_2007}
and fruit flies\cite{shaw_stress_2002}.

In all but one tested models, sleep deprivation led to death.
However, the physiological explanations are, to date, unclear.
In both rats and dog pups, death was linked to dramatic metabolic changes and obvious signs of distress,
altogether suggesting that mortality could result from a generalised and confounding effect of the undergone stress rather than the sleep deprivation itself\cite{bentivoglio_pioneering_1997, rechtschaffen_sleep_1995}.
The same chronic sleep deprivation paradigm that had been reported to kill rats did not kill pigeons\cite{newman_sleep_2008}. 
In the Pacific beetle
cockroach (\emph{Diploptera punctata}), sleep deprivation was performed by continuously stimulating the insects\cite{stephenson_prolonged_2007}, but did not account for exhaustion-induced stress\emd{}which is known to be lethal to other species of cockroaches\cite{cook_neurophysiological_1974}.
Finally, in \emph{Drosophila} the evidence is limited to one study that observed partial lethality in a very small number of wild-type flies (4 of the initial 12 individuals died after 70~h),
which the authors sleep deprived manually by tapping the insects' tube with their fingers\cite{shaw_stress_2002}.

An alternative approach to address this question is to study the trade-off between sleep and fitness by asking whether sleep amounts correlate with, for instance, lifespan. 
Humans affected by fatal familial insomnia, a rare condition, are known to live only a few months after the onset of the disease\cite{schenkein_self_2006}.
In fruit flies, some mutants that have reduced sleep incidentally die faster\cite{cirelli_reduced_2005}, but others do not\cite{kume_dopamine_2005}.
Such correlations have, however, been more difficult to characterise in non-pathological cases.
For instance, in humans, epidemiological studies have shown that low, but also high, sleep amounts are associated with increased mortality (\ie{} a U-shape relationship)\cite{akerstedt_sleep_2017,akerstedt_sleep_2018}, but there is no consensus on the matter\cite{kurina_sleep_2013}.
In fruit flies, a recent study artificially selected populations according to their amount of activity, and showed that, even though sleep amount is highly heritable, selected low and high sleepers had the same lifespan as the control populations\cite{harbison_selection_2017}.

\subsubsection{Energy conservation}
Perhaps the most obvious general function that sleep could have is simply not to waste energy being active.
Indeed, within an ecosystem, organisms tend to adapt to partition their respective niches, one example being temporal division.
Namely, most animals are either nocturnal, diurnal or crepuscular, with various degrees of specialisation.
A consequence of temporal partitioning of ecological niches is that the physiology of organisms is tuned to a specific range of abiotic environment that vary over 24~hours (\eg{} light, humidity and temperature).
In addition, their biotic environment may oscillate too.
For instance, the parasite and predation pressure, as well as the availability of food, have their own daily dynamics. 
In this context, it is advantageous to minimise risk and expend the least possible energy whilst waiting for the time of the day with the most favourable conditions\cite{mignot_why_2008}.

Following this principle, sleep would have evolved as\emd{}and, to a large extent, would still be\emd{} a daily state of dormancy or torpor\cite{siegel_sleep_2009}.
In addition to daily oscillations, there are ecologically relevant rhythms along which many animals engage in reversible states of immobility.
At low tide, many marine invertebrates become immobile which both protects them and saves their energy\cite{mcmahon_respiratory_1988,dahlhoff_physiological_2002,connor_high-resolution_2011}.
Likewise, some species respond to seasonal variation by entering states such as hibernation or æstivation, which, in many respects, resembles sleep\cite{siegel_sleep_2009}. 
In temperate and dry climates, for instance, land snails are mostly dormant during the summer with a greatly reduced overall metabolism\cite{wunnenberg_diurnal_1991, rahman_consequences_2012}, but can revert to an active state when and if precipitations occur. 
Over evolutionary times, organisms that developed adaptations to seasonal variations may eventually depend on them, which could mean they have become homeostatically controlled.
For instance, many plants absolutely need periods of dormancy to germinate or flower\cite{milberg_does_1998,bentsink_seed_2008}.
Similarly, sleep could have evolved from an opportunity to a necessity. 
In brief, under the energy conservation hypothesis, sleep is primarily and mostly a daily state of torpor that in some, but not all, taxa would have been co-opted to support a variety of other physiological processes.

\subsubsection{Restoration}
\label{sec:restoration}
Alternatively, it has been suggested that sleep primarily serves a restoration purpose whereby the physiological and metabolic expenditures that happened during the day, can be counterbalanced by a period of rest during the night\cite{adam_sleep_1980} (for nocturnal animals).
There are several examples of such `active' recuperative processes, many of them relating to brain functions, but also several showing the implication of sleep in the replenishment of other physiological systems.

\paragraph*{Brain functions}
Prolonged wakefulness has been shown to results in the accumulation of reactive oxygen species,
causing oxidative stress that explains some of the effects of sleep deprivation\cite{ramanathan_sleep_2002}.
Similarly, a notable study in rats suggested that sleep was linked with an increased interstitial space between brain cells, which could be crucial in order to clear toxic by-products that accumulate during wakefulness\cite{xie_sleep_2013}. 

\paragraph*{Distributed functions}
The immune system has also been linked to sleep in two ways (reviewed in \cite{bryant_sick_2004,imeri_how_2009}).
Firstly, sleep is altered during infections.
For instance, in rabbits, the infection by various micro-organisms causes an increase in sleep\cite{toth_alteration_1988}.
This also happens during the response to non-infectious components of the pathogens, such as lipopolysaccharide\cite{lancel_lipopolysaccharide_1995}.
Secondly, sleep deprivation compromises immune competence.
For instance, it was shown that partially sleep-deprived humans immunised against the flu had 
subsequently fewer antibodies\cite{spiegel_effect_2002}.
However, the conclusions are not consensual, as some studies show, instead, a protective effect of sleep deprivation on immune functions\cite{renegar_progression_2000}.
Suggesting non-linear effects of sleep loss on immunity\cite{bryant_sick_2004}.


Beyond immunity, sleep has been suggested to take part in other distributed processes.
In rats, it is thought to be involved in efficient wound healing\cite{gumustekin_effects_2004}.
Sleep has also been presented as a mechanism to restore muscles\cite{dattilo_sleep_2011}.
Recently, an interesting study in mice proposed that the skeletal muscles could be implied in its regulation, suggesting a crucial role in muscular recovery\cite{ehlen_bmal1_2017}.


\subsubsection{Learning and memory}

One of the most compelling putative function of sleep is its role in brain plasticity.
Indeed, most animals with a central nervous system are capable of some sort of learning, the flip side of which could be the burden of sleeping.
The idea that sleep plays a crucial role in the formation of memories is relatively old\cite{jenkins_obliviscence_1924}.
It has since been supported by a large corpus of experimental work on various models (reviewed in \cite{stickgold_sleep-dependent_2005}).
For example, spatial memory is impaired in sleep-deprived rats\cite{smith_evidence_1996},
Some forms of learning, such as the visual discrimination task, requires sleep after training\cite{stickgold_visual_2000}.
In human subjects, the learning of motor sequences is greatly improved after a night of sleep\cite{walker_practice_2002}, but not if subjects are kept awake\cite{walker_sleep_2003}.

The mechanism by which sleep affects memory is, however, unclear and two apparently contradictory theoretical 
frameworks are currently debated: the \emph{active system consolidation hypothesis} and the \emph{synaptic homeostasis hypothesis} (reviewed in \cite{diekelmann_memory_2010} and\cite{tononi_sleep_2014}, respectively).

\paragraph*{Active system consolidation hypothesis}
It states that, during wake, events are perceived and encoded in the brain in a shallow fashion.
During sleep, the nervous system would `replay' these events and the connections would be strengthened and redistributed\cite{mcclelland_why_1995,marshall_contribution_2007}.
This would allow brains to form new, persistent, representations that are more symbolic and abstract\cite{rasch_maintaining_2007,diekelmann_memory_2010}.

In mammals, there is evidence that such process happens during \gls{sws} in particular, highlighting the interplay between the hippocampus and the neocortical regions\cite{rasch_odor_2007,girardeau_selective_2009}.
Behaviourally, it also accounts for the observation that sleep qualitatively changes memory, allowing for new representations\cite{ellenbogen_human_2007,wagner_sleep_2004}.

\paragraph*{Synaptic homeostasis hypothesis}
A relatively recent theoretical framework that associates sleep to brain plasticity is the synaptic homeostasis hypothesis\cite{tononi_sleep_2003,tononi_sleep_2006}.
It accounts for sleep as the `price to pay for plasticity'.
It is built on the notion that learning generally involves gathering input from the environment, which happens during wake\emd{}as the nervous systems is connected to the outside world.
It postulates that, during wake, synapses between neurons are potentiated (\ie{} strengthened), hence creating associations and memories.
Then, during sleep, the brain is essentially off-line, which allows it to sample comprehensively the links previously drawn.
In this process, synapses that are not strong enough may be down-selected (\ie{} downscaling) in order to preserve only robust associations\cite{tononi_sleep_2014}.
In addition to reducing the noise generated during wake, downscaling would also promote the encoding of memory in subsequent wake periods.

The synaptic homeostasis hypothesis has been corroborated at several levels\emd{}at least in mammals (reviewed in\cite{tononi_sleep_2014}). 
Firstly, molecularly, proteins that are involved in potentiating synapses are more abundant and active during wakefulness\cite{vyazovskiy_molecular_2008,lante_removal_2011},
Secondly, several studies have excited upstream neurons during sleep and measured the post-synaptic response, showing that, after sleep, synapses were globally weaker\cite{vyazovskiy_molecular_2008,vyazovskiy_cortical_2009,liu_direct_2010}.
Lastly, structurally, it has been observed that mice neurons have more dendritic connections during wake, and that their number reduce after sleep\cite{maret_sleep_2011}.
%cite{driver_daf-16/foxo_2013}

\subsection{Regulation of sleep}
\label{sec:two-process}
Sleep is generally postulated to be regulated both by the internal clock (process $C$) 
and by the homeostat (process $S$).
This view has been formalised Alexander Borbély as the `two-process model of sleep regulation' (fig.~\ref{fig:two-processes})\cite{borbely_two_1982,achermann_two-process_2004,skeldon_mathematical_2014}.
In its original form, it states that the propensity to sleep is the sum of the effect of $C$ and $S$.
The first arm, $C$, is a periodic function of time and independent of the occurrence of sleep.
The second one, $S$, represents the accumulation of sleep pressure in an animal that remains awake and is, therefore, a conditional function of the realisation of past sleep.
When sleep happens, $S$ decreases exponentially.
On the other hand, when sleep cannot occur, it increases asymptotically.
$S+C$ defines the overall sleep pressure that, above a certain threshold $H^+$, triggers sleep.
Conversely, when $S+C$ becomes lower than $H^-$, the animal wakes up.


\begin{figure}[h!]
  \centering   
   \includegraphics[width=0.95\textwidth]{\currfiledir/.\currfilebase.pdf}
  \caption[The two-processes model of sleep regulation]{\ctit{The two-processes model of sleep regulation.}
	Theoritical example illustrating the model for a diurnal animal, over three days, in two scenarios:
	either it is allowed to sleep throughout (\textbf{A}), or it is prevented to do so during the first night (\ie{} sleep deprivation, red background, \textbf{B}).
	The propensity to sleep is the sum of two processes: $C$ and $S$. $C$, the clock, is a periodic function that does not depend on sleep pressure.
	$S$, represents the homeostat. It decreases when sleep occurs (blue), and increases when it does not.
	Sleep occurs when $C+S > H^+$, and an animal wakes up when $C+S < H^-$.
  	\textbf{A}, In normal circumstances, $C$ and $S$ add up to a periodic function that is the sleep propensity, and sleep is the realisation of such propensity.
  	\textbf{B}, During sleep deprivation, the clock ($C$) is unaltered, but the sleep homeostat ($C$) increases asymptotically. 
	When sleep deprivation is released, the animal may still have an overall sleep pressure ($C+S$) greater than ($H^+$) and will dissipate $S$ exponentially.
	In this example, there is such a sleep rebound, during the day, immediately after sleep deprivation.
  \label{fig:\currfilebase}
  }
\end{figure}








Under normal conditions, when an animal is allowed to sleep, process $S$ synchronises with $C$
to produce a regular binary sleep pattern (fig.~\ref{fig:two-processes}A).
When sleep cannot occur\emd{}for instance, because an external factor prevents it\emd{}process $S$ continues to increase until it can dissipate (fig.~\ref{fig:two-processes}B). 
Interestingly, in some cases the recovery of $S$ (sleep rebound) can be only partial as process $C$ may wake an animal before it has fully recovered. 
Under the two-process model, this may result in an earlier onset of subsequent sleep.
If this model has served as a major conceptual framework since its postulation, in the 1980s, it has also been refined\cite{borbely_two-process_2016} and adapted\cite{skeldon_mathematical_2014}.
%One of the main consideration has been that its original form did not account for the fact that many animals have polyphasic sleep instead of one continuous sleep bout each day, which implies some level of dissociation between the two processes.

In conclusion to this general introduction, the open definition of sleep has allowed is discovery and investigation in many species, which has led to the view that sleep its a ubiquitous behaviour.
In the next two sections, I will describe why \dmel{} has emerged as a powerful model to unravel the mechanistic aspects of behaviour, and how it became instrumental in studying sleep.

\clearpage
\section{\dmel{}, a toolbox to study behaviours}
\label{sec:droso-as-model}

In order to fully grasp how and why the fruit fly became a significant model to study sleep,
I believe it is necessary to first see how it emerged as a powerful tool to study the genetics of behaviour in general.
In this section, I will briefly present, for the reader who may not be familiar with it, general aspects of the biology of \dmel\emd{}the interested one will easily find much better written historical, practical and fundamental material, for instance \cite{greenspan_fly_2004,mohr_first_2018,markow_natural_2015}.
Then, I will explain specifically the context in which it developed as a model for behaviour.
Finally, I will describe some of the methodological resources that have emerged in the field.

\subsection{Life history of \dmel{}}

\dmel{} is a 3~mm long insect belonging to the Diptera order\emd{}a taxum with more than $100,000$ species of insects, including all flies and mosquitoes\cite{mayhew_why_2007}. 
The genus \emph{Drosophila} itself contains more than 1500 species\cite{pelandakis_molecular_1993}. 

The original habitat of \emph{D.~melanogaster} is thought to be in sub-Saharan Africa from which it would have dispersed to Europe and Asia only $10,000$ to $15,000$ years ago, and more recently to America and Australia. 
The expansion of its habitat is thought to be related to human migrations\cite{markow_natural_2015}.

Like other Dipera, fruit flies are holometabolous organisms (\ie{} with complete metamorphosis) whose life history is characterised by an asexual larval stage that has a different ecological niche and morphology from the adult.
Adult females lay up to 100 eggs per day in rotting fruits in which their larvae will hatch and develop.
\dmel{} larvae, like other arthropods', grow discontinuously through successive moulting events.
In the case of fruit flies, there are three larval stages (\ie{} instars).
It takes about four days (at 25$^{\circ}$C, with unlimited food supply) for larvae to reach the end of their third, and last, instar, after which they
crawl out of their substrate and initiate pupariation.
During pupariation, which lasts approximately four days in \dmel, the insect undergoes dramatic anatomical and physiological changes. 
When the metamorphosis is complete, the insect's puparium splits, and the imago (\ie{} the final adult form) emerges.
Adult flies live up to 90 days at 25$^{\circ}$C, in the laboratory, though their lifespan in the wild is thought to be shorter\cite{wit_laboratory_2013}.

%The ecology of  \emph{D.~melanogaster} is sadly not extensively studied, and the strains used in laboratories are highly domesticated animals that are likely adapted to laboratory conditions.
%sexual dimorphism,

\subsection{A century of research on the fruit fly}

In the beginning of the 20\textsuperscript{th} century, the principles of heredity, and their link to the theory of natural selection, were 
a matter of great interest and controversy, but an outlier group, led by Thomas Hunt Morgan, was to revolutionise the field of genetics with the help of a tiny insect: the fruit fly \dmel{}.

\dmel{} had been described by Johann Meigen in 1830, and was studied by several groups.
In particular, Frank Lutz was using it to perform experimental evolution, in a Darwinian framework.
It is thought to be Lutz who introduced Morgan to the fruit flies.
Morgan eventually noticed heritable changes, such as the white eye mutation, and observed that their transmission was non-Mendelian: instead, some heritable traits were linked to one another, which corroborated a chromosomal theory of heredity\cite{carlson_how_2013}.


His seminal work initiated a century of fruitful work using \droso, which was punctuated by several Nobel prizes, staring by Morgan himself in 1933.
Later, Morgan's former student, Herman Muller, developed a technique to generate random mutations using X-rays which was also awarded the prize in 1946.
The use of the \droso{} model then propagated to other fields than genetics.
In 1995, Edward Lewis, Christiane Nüsslein-Volhard and Eric Wieschaus, were credited for their use of the fruit fly to unravel the genetics of embryonic development\cite{nusslein-volhard_determination_1987}.
In 2011 the prize was shared by Jules Hoffman for the work he and his team carried on innate immunity\cite{lemaitre_dorsoventral_1996, oneill_history_2013}.
Altogether, by the mid-twentieth century, it made no doubt that \emph{Drosophila} was an instrumental tool to study the genetic underpinnings of physiological and developmental processes.

The extent to which single genes could determine much more `complex' phenotypes such as behaviours was, however, vividly debated\cite{sokolowski_drosophila_2001,tully_discovery_1996}.
In the late 1960s Seymour Benzer reasoned that specific behaviours could be investigated using the same principles as for other traits, and he designed a simple quantitative assay to score light preference in a population of flies.
He then carried a genetic screen and discovered mutants with altered phototaxis and isolated the genes responsible for it\cite{benzer_behavioral_1967}, hence demonstrating genetics could be used to address the molecular basis of behaviour\cite{benzer_genetic_1973,bellen_100_2010}.
In the early 1970s, Benzer and his team used the same approach and created an assay to score eclosion time and locomotion which led to the discovery of the \emph{period} gene, a core gear of the circadian clock.
Following Benzer's footsteps, in 2017, Jeffrey Hall, Michael Rosbash and Michael Young were awarded the Nobel prize for their work of the molecular mechanisms of the circadian clock\cite{konopka_clock_1971,reddy_molecular_1984,bargiello_restoration_1984,zehring_p-element_1984,bargiello_molecular_1984}.


By the end of the twentieth century, two lessons had been learnt.
Firstly, the research on fruit flies had provided the community with a formidable set of tools to study of how genes can determine even complex phenotypes.
Secondly,  the molecular determinism of ubiquitous processes, such as development, immunity and circadian clocks could then be generalised and translated to other organisms.
In the early 2000s, when the community needed a simple model to study the genetics and neuronal basis of sleep\cite{hendricks_need_2000}, the fruit fly emerged as a natural candidate\cite{hendricks_rest_2000, shaw_correlates_2000}.

A large part (the next two chapters) of the work I present in the thesis herein is methodological.
Therefore I will use this section, to specifically review the methods developed to study behaviours (in particular sleep behaviour) in \droso{}, 
and will dedicate the next subsection (\ref{sec:sleep-droso}) to the discoveries resulting from their application to sleep during the last two decades.

%Firstly, I will briefly describe the biology of \dmel{} itself as a model organism.
%Secondly, I will present it as tool for studying behaviour in general.
%Finally, I will describe the set of methods and paradigms used to study specifically the sleep behaviour.

\subsection{Study of behaviour in \droso{}}

\subsubsection{What are behaviours}
It is complicated to formally define behaviour in the context of biology, and\emd{}implicit or explicit\emd{}definitions vary between authors\cite{reiser_ethomics_2009,anderson_toward_2014,egnor_computational_2016, brown_ethology_2018, berman_measuring_2018}. 
Here, I will define behaviour in a broad and computational sense as 
a time series of observable states that are organised in a hierarchical fashion.
States can be, for instance, postures, sounds, visual displays, chemical signals or any combinations of these.
They can vary over time and may be either discrete (within a repertory) or continuous (bounded to a range).
Behavioural states are hierarchical insofar as a state may be defined by a time-series of sub-states\cite{berman_measuring_2018}.
In this respect, states are contained in states of higher order, much like linguistic entities (\eg{} syllables 
make words and words make sentences)\cite{richard_hierachical_1976, dawkins1976hierarchical,berman_predictability_2016,gomez-marin_hierarchical_2016}.

To illustrate this hierarchical organisation, one can think of the courtship behaviour in fruit flies.
Throughout its life, a male may engage in different states such as feeding, resting, flying, walking, grooming or courting a female. 
These are behaviours of high order, which are observable over a long timescale.
However, in our example, the courting behaviour is itself made of a series of lower order behaviours such as chasing, wing extension or wing vibration (see \cite{lasbleiz_courtship_2006} for a review).
Then, at an even shorter timescale, a state such as `wing vibration' can be seen as a series of individual singing bouts, and each bout as a series of pulses\emd{}each level having its own `grammar'\cite{arthur_multi-channel_2013}.

In the next two subsections, I will start by describing a few landmark studies to illustrate how the interest in behavioural genetics preceded by far the application of high-throughput methods to it.
Then, I will show how, in recent years, ethology has increasingly relied on high-throughput data acquisition and data sciences.
Even though the study of behaviour goes far beyond \droso\emd{}with lots of laboratory work on mice, \emph{C.~elegans}, zebrafish and many more\emd{}, I will narrow the scope of this introduction to the work on the fruit fly.

\subsubsection{Predigital era}

During most of the 20\textsuperscript{th} century, it was not possible, or straightforward, to digitalise data.
Benzer, whose early work I just described, was not the only researcher who had to invent ingenious paradigms to score high-level behaviours. Beyond circadian activity and phototaxis, behaviours such as courtship, foraging,  egg laying or even, for instance, learning were under scrutiny.

Courtship is a stereotyped behaviour that is generally directed from males to females, and that is central to a male's fitness\cite{greenspan_courtship_2000}. As such, it has been extensively studied.
It was noticed that this fundamental behaviour could be altered in the \emph{fruitless} mutants\cite{gill_mutation_1963,hall_courtship_1978}.
In particular, it was observed that, in this genotype, males courted other males, sometimes forming `chains'.
Courtship is subdivided in several parts which were\emd{}and often still are\emd{}scored through subjective observation.
Interestingly, in order to increase throughput and decrease subjectivity, assays such as the `chaining index', which scores the proportion of males engaged in courting chains, were proposed\cite{villella_extended_1997,demir_fruitless_2005}


In 1980, Marla Sokolowski published her discovery of a naturally occurring difference of foraging behaviour in the larva.
In the presence of patchy food, she noticed that some animals, the `rovers', were consistently covering a large distance whilst others, the `sitters', had a lower propensity to leave their food patch.
This foraging behaviour could be quantified with an elegant assay as larvae leave a visible trail\emd{}whose length can be approximated\emd{}on their substrate. The molecular  determinism of this trait was then identified\cite{osborne_natural_1997}, and is considered `the first example of the molecular identification of a naturally occurring behavioural variation'\cite{sokolowski_drosophila_2001}.

%\todo{learning Dunce}
%\todo{egg laying}
\subsubsection{Ethomics, a high-throughput approach to behaviour}
\label{sec:tracking-tools}
In the last few decades, our capacity to record and process large quantities of scientific data has tremendously increased\cite{stephens_big_2015}, and biology is undergoing a transition towards data sciences.
These recent developments have virtually affected all fields from
structural biology\cite{levitt_birth_2001-3} and 
genetics\cite{schatz_biological_2015} to
neurosciences\cite{sejnowski_putting_2014} and
ecology\cite{pascual_computational_2005}.


The study of animal behaviour is also undergoing its own computational transition,
which has prompted the terms `ethomics'\cite{branson_high-throughput_2009, reiser_ethomics_2009} and `computational ethology'\cite{anderson_toward_2014}. 
It is now accepted that ethology too can benefit from quantitative sciences such as machine learning, physics and computational linguistics\cite{gomez-marin_big_2014,egnor_computational_2016,brown_ethology_2018, berman_measuring_2018}.
There are two types of interdependent technological breakthroughs that underlie the current transition.
Firstly the availability of new high-throughput data acquisition methods.
Secondly, the application of computer vision, statistics and the computational techniques to the processing of the resulting data.

In the context of high-throughput biology, data acquisition techniques involve recoding
physical phenomena as \emph{digital} time series. 
There is a variety of approaches ranging from building a specific hardware that records directly variables that are biologically meaningful, to using general purpose recording equipment and algorithmically interpreting the primary data into meaningful secondary data.

One of the most widely used specific hardware is the \gls{dam} system (Trikinetics Inc.) that I will describe in detail in subsection~\ref{sec:immob} which is used to score walking activity.
Two recent methods have been proposed to monitor feeding behaviour using the electrical signal generated when the fly touches its food\cite{itskov_automated_2014,ro_flic_2014}.
Jamey Kain and co-workers described an advanced system designed to track the movement of all six legs on a fly walking on a ball, and analyse behaviour in real time, with he possibility of performing closed loop experiments (\eg{} deliver stimuli when the animal turns right)\cite{kain_leg-tracking_2013}.
Karl Gotz designed a system to record the amplitude of wingbeats in a tethered fly\cite{gotz_course-control_1987}.
In a series of experiments, the group of Michael Dickinson then used this system, in addition to the recording of several flight variables (\eg{} pitch, roll and yaw) in a closed-loop manner to study response to visual signals\cite{sherman_comparison_2003,maimon_simple_2008}.

The other approach is to use off-the-shelf tools such as microphones and cameras.
For instance, Benjamin Arthur and collaborators have used an array of microphones to record the courtship songs of 32 males at a time, for several consecutive hours\cite{arthur_multi-channel_2013}. They then used signal processing techniques to extract meaningful acoustic secondary data.

In recent years, cameras and computer power have become less expensive.
Therefore, there have been countless applications of video tracking to the study of fly behaviours.
From the 1990s already, video analysis had been used, for instance, to study whether the locomotor activity
was organised in a fractal fashion\cite{cole_fractal_1995} and to the dynamic of trajectories\cite{martin_portrait_2004}.
Various implementations of video monitoring have been used for sleep research 
to score immobility of many animals over several days\cite{zimmerman_video_2008,gilestro_video_2012, donelson_high-resolution_2012,faville_how_2015} (see subsection~\ref{sec:immob}).
Interestingly, setups using multiple cameras have been used to reconstruct flight trajectories\cite{straw_multi-camera_2010}.

Rather than focusing on one individual, some paradigms have been designed to automatically score interactions between two animals,
for instance aggression\cite{dankert_automated_2009} and courtship\cite{tsai_image_2012}.
Furthermore, some software tools were designed to track multiple animals in a social context\cite{branson_high-throughput_2009,perez-escudero_idtracker_2014,kabra_jaaba_2013,robie_mapping_2017}.
In the last year, algorithms using deep learning have been proposed to track unmarked animals, whilst minimising identity loss\cite{romero-ferrero_idtracker.ai_2018} and to detect body posture\cite{pereira_fast_2018}.

Noticeably, some methods have also been proposed to video track larvae, which are translucent\cite{pistori_mice_2010,sinadinos_increased_2012,swierczek_high-throughput_2011}. 
Recent developments include the use of frustrated total internal reflection to increase optical contrast\cite{risse_fim_2013,risse_quantifying_2015}.

In general, the low-level representations (\eg{} pixel and position) that are generated by recording equipment (or by algorithms) need to be processed further in order to make biological sense, and there is also an ongoing effort to develop methodological and conceptual frameworks in this direction\cite{brown_ethology_2018, berman_measuring_2018}.

For instance, Joshua Vogelstein and colleagues applied multiscale unsupervised structure
learning\cite{rockmore_modern_2004} to hierarchically cluster larval behavioural time-series, and studied the genetic determinism of
these complex phenotypes\cite{vogelstein_discovery_2014}.
Another study, by  Berman \emph{et al.}, proposed a pipeline in which the repertoire of behaviours is represented as a behavioural space\cite{berman_mapping_2014}.
The authors first used spectral features to generate a high-dimensional space, and then applied t-SNE\cite{maaten_visualizing_2008} to embed feature vectors and cluster states.
In a follow-up article, they develop the idea of transitions and hierarchy in the behavioural space\cite{berman_predictability_2016}.


In conclusion to this section, \droso{} has proved as a powerful tool to study the genetic determinism of behaviour.
In recent years, technological and conceptual developments have promoted it at the forefront of neurobiology and ethomics.
In the next section, I will describe how, from this substrate, \droso{} emerged as a model for sleep.


\clearpage
\section{Sleep in \dmel}
\label{sec:sleep-droso}

\subsection{Generalities}

The `droso' in `\droso{}' means{} `dew', reflecting its preference for dawn and dusk, when dew condensates. 
Indeed, it has long been known to be crepuscular (\ie{} active during the twilight).
It is consensual that fruit flies have an evening and a morning peak of activity, 
centred on the light transition time (fig.~\ref{fig:sleep-droso-exple}).
Conversely, sleep is thought to peak in the middle of the day (\ie{} `siesta') and night.


\begin{figure}[h!]
  \centering   
   \includegraphics[width=0.95\textwidth]{\currfiledir/.\currfilebase.pdf}
  \caption[Example of sleep-time visualisation]{\ctit{Example of sleep-time visualisation.}
	Simplified `sleep trace' representing the probability of observing a sleeping fly, $P(asleep)$, over time.
	This example represents the usual behaviour of a male population.
	Time is often expressed as Zeitgeber Time (ZT) which is $\text{ZT} = Time \pmod{24~h}$, and zero corresponds to the start of the day (phase L).
	In this example, the light regime is 12~h of light and 12~h of darkness (12:12~h, L:D).
	Males tend to have a bimodal activity  with a morning and an evening peak, and are inactive in between.
  \label{fig:\currfilebase}
  }
\end{figure}








The dynamic of sleep is greatly affected by biotic and abiotic factors, but also by how sleep is defined and scored. 
In this section, I will first detail the methods 
that have been used to characterise, score and perturb sleep in \droso, and their recent development.
Then, I will describe how internal and external variables impact sleep.
Finally, I will present the putative functions of sleep in fruit flies.

\subsection{Methods and paradigms to study sleep}

Methods often frame and drive scientific research, but have also been argued to hold epistemic value and, as such, they impact the nature of scientific conclusions in a given field\cite{baird_thing_2004}.
The behavioural definition of sleep I presented in subsection~\ref{sec:sleep-def} relies on three pillars (\ie{} quiescence, homeostasis and arousal),
which prompts the need for three corresponding methods:
the ability to \emph{score immobility}, the possibility to \emph{sleep-deprive} flies and tools to \emph{probe arousal}.
In this section, I will describe the tools and paradigms that have been proposed to address each of these three needs.

\subsubsection{Immobility scoring tool}

\label{sec:immob}

\paragraph*{Early work and manual scoring}
In 2000, two seminal studies addressed the question of sleep in the fruit fly\cite{hendricks_rest_2000,shaw_correlates_2000}.
Hendricks and colleagues performed an \gls{ir} beam cross locomotion assay (see fig.~\ref{fig:dam}) on a small number of animals ($N=11$), but crucially also videotaped the flies\cite{hendricks_rest_2000}.
They noticed that flies engaged in long bouts of inactivity in proximity with, but not on, the food and during such immobility bouts, animals adopted a specific posture.
They then videotaped a population of flies in a Petri dish and scored, manually, their immobility.
The authors noticed a satisfactory correspondence between the manual scoring from video data and the \gls{ir} assay.
Therefore, in order to automatise behaviour quantification and, ultimately, increase throughput, they continued  with the latter assay.

The second study used an ultrasound system to perform high-resolution measurement of fine movements\cite{shaw_correlates_2000}.
The authors first ensured their system matched visual observations.
Then, in order to work with a larger number of flies, they adopted the same \gls{ir} assay as\cite{hendricks_rest_2000}.
They validated this decision by recording 7~flies for 18~h, only finding 2.35\% false negative of motion (animals that were scored as moving by the ultrasound system, but not crossing the \gls{ir} beam).



\paragraph*{The \acrfull{dam}}
\label{par:dam}
The \gls{dam} system (TriKinetics Inc.) is a tool to measure the activity of fruit flies which was primarily designed for the study of the circadian rhythm\cite{hamblen_germ-line_1986}.
A single device (fig.~\ref{fig:dam}A)
can record the movements of up to 32 animals over long durations.
Each fly is contained in a narrow glass tube along which it walks.
When an animal crosses the midline of its tube, the device detects it through an \gls{ir} beam and sensor (fig.~\ref{fig:dam}B).
Because of its prior availability, robustness and scalability, this system has met large adoption in the field and, to date, remains the most widely used tool to score sleep in fruit flies.

\begin{figure}[h!]
	\includegraphics[width=0.95\textwidth]{\currfiledir/.\currfilebase.pdf}

	\caption[Drosophila Activity Monitor]{\ctit{Drosophila Activity Monitor.}
		\textbf{A}: a Drosophila Activity Monitor (DAM), the most widely used system to monitor walking activity and sleep in \droso, with 8 tubes loaded.
		Each monitor can hold up to 32 single flies in glass tubes.
		Multiple monitors can run in parallel, and their data is centralised on an acquisition computer.
		\textbf{B}: Schematic of a single tube ($\approx 70$~mm long) showing the underlying mechanism of the DAM.
		Each fly is contained between food on one side and cotton wool on the other. 
		Tubes are sealed with paraffin wax.
		When the fly walks through the midline of its tube, it crosses an infrared beam which is detected as movement.
		The numbers of beam crosses are generally aggregated over one minute (\ie{} the resulting sampling rate being 1~min$^{-1}$).
		\label{fig:\currfilebase}
	}
\end{figure}


The \gls{dam} paradigm cannot exclude the possibility that an animal could be very active without crossing the midline of its tube.
For instance, a fly could walk and turn before reaching the tube's midline, or spend a large fraction of time eating or grooming, which would result in falsely scoring immobility and ultimately overestimating sleep\cite{zwaka_context_2015}.
In order to reduce the possibility of such low amplitude movement being unnoticed, TriKinetics also developed a `multibeam' system (DAM-5), 
which is an upgraded version with 17~beams. 
However, it is less cost and space effective and has therefore only been used in a few sleep studies
(for instance, \cite{garbe_context-specific_2015,dilley_behavioral_2018}).

\paragraph*{Video tracking} Another solution to address the issue of movement false negatives is to video track flies.
Various authors, including myself, have contributed in this direction\cite{zimmerman_video_2008,donelson_high-resolution_2012,gilestro_video_2012,faville_how_2015,murphy_postprandial_2016,geissmann_ethoscopes_2017}, but there seems to be little overall adoption in the field despite the availability of a wide panel of tools.

Zimmerman and colleagues developed a \texttt{Matlab~\&~C++} software specifically to score sleep in \droso{}\cite{zimmerman_video_2008}.
Using simple background subtraction and binary threshold, they were able to show that, under some circumstances (sex, age, genotype and time of the day),
\gls{dam} critically overestimated sleep.
Donelson \emph{et al.} wrote `Tracker' (a \texttt{java} program) also to quantify sleep and confirmed these findings\cite{donelson_high-resolution_2012}.
A video-based tracking tool was also developed by my supervisor, Giorgio Gilestro, with emphasis on scalability\cite{gilestro_video_2012}.
Ethovision, a commercial software with a broader scope, has also been used in some studies\cite{garbe_context-specific_2015,cavanaugh_drosophila_2016}.
`DART', a \texttt{Matlab} program was designed as a solution to track and stimulate flies in the context of sleep\cite{faville_how_2015}.
Recently, Murphy and co-workers developed `ARC', a system to score activity and feeding at the same time\cite{murphy_postprandial_2016}.

\paragraph*{Electrophysiological attempts}

%Several studies have also attempted to quantify sleep using non-behavioural read-outs, such as electrophysiological ones, but were unfortunately not very compelling
There have been several attempts to find electrophysiological correlates of sleep behaviours in \droso{} by measuring the local field potential of tethered flies\cite{nitz_electrophysiological_2002,van_swinderen_uncoupling_2004,van_alphen_dynamic_2013}.
In addition, calcium imaging\cite{bushey_sleep-_2015} and single neurone recording\cite{pimentel_operation_2016} have been used to compare neuronal activity during sleep or wake.
However, the evidence for an unambiguous read-out for sleep remains unclear\emd{}and the setup too impractical\emd{}to justify the wide adoption of this method.




\paragraph*{The `five-minutes rule'}
The first two studies in the field reported that startled wild-type flies were less likely to respond if they had been quiescent for five minutes or more\cite{hendricks_rest_2000,shaw_correlates_2000}.
Later, another study corroborated this observation\cite{huber_sleep_2004}, which has led most authors to generalise this finding to define sleep as, at least, five consecutive minutes of immobility\emd{}a rule now widely used.

For instance, in some of the early landmark papers:
`sleep was measured as bouts of 5~min of inactivity, as described previously'\cite{joiner_sleep_2006}, %mutants
`a sleep episode was defined as a 5-min bin of uninterrupted quiescence'\cite{pitman_dynamic_2006},
`sleep is defined here, as in previous work, as behavioural immobility lasting 5~min or more'\cite{cirelli_reduced_2005} and
`5~min intervals without any locomotor activity'\cite{kume_dopamine_2005}.

However other studies have shown that responsiveness is not constant, but follows a temporal dynamic that may vary. 
In other words, it seems that arousal threshold is not a binary\emd{}or even a monotonic\emd{}function of the time spent immobile,
and that it depends on other variables such as time of the day, sex and genotype\cite{van_alphen_dynamic_2013,faville_how_2015}.


\subsubsection{Sleep deprivation tools}
\label{sec:sleep-dep-tools}
Several paradigms have been proposed to keep animals awake for various durations.
They generally include the delivery of recurrent mechanical stimuli\cite{shaw_stress_2002, huber_sleep_2004,sauer_sleep_2004,li_sleep_2009,linford_re-patterning_2012, faville_how_2015}, but also
hypnotic and stimulant drugs\cite{hendricks_modafinil_2003,andretic_dopaminergic_2005},
exposure to conspecifics\cite{gilestro_widespread_2009,beckwith_regulation_2017}, 
olfactory cues (French \emph{et al.}, in prep.) and
%optogenetic and 
thermogenetics\cite{seidner_identification_2015,dubowy_genetic_2016}.


%For this reason, classically, in fruit fly, \gls{sd} is performed mechanically;  by forcing the animal to keep moving \cite{huber_sleep_2004}, using repeated mechanical stimulations at periodic or random intervals. 
%There seem to be no clear consensus on the intensity and frequency of stimulus delivery, but they are often intense rotatory movement from a vortex\cite{linford_re-patterning_2012}, shaker\cite{li_sleep_2009} or a device that linearly impacts tubes\cite{huber_sleep_2004}.
%Mechanical disturbance are generally delivered simultaneously to groups of fruit flies at a frequency of one to three per minute.
%Interestingly, recent research shows that it seems possible to thermogenetically sleep deprive fruit flies\cite{dubowy_genetic_2016}.


\paragraph*{Manual sleep deprivation} 
In the first study reporting sleep in \droso{}, the authors started by sleep depriving flies manually\cite{hendricks_rest_2000}.
This allowed them to tune the timing of the stimulus delivery \emph{dynamically}\emd{}that is, according to whether flies were immobile.
Specifically, `mechanical stimuli were applied whenever $\ge$~1 fly was immobile for $\ge$~1 min'.
In addition, they were able to deliver stimuli of appropriate strength:
`a more intense stimulus was applied after 15~s if needed.
The stimuli were graded as 1 (one tap), 2 (two taps), 3 (move dish 1~mm)
and 4 (lift dish and tap forcefully)'.

A similar approach was used by Shaw \emph{et al.} in the second seminal study\cite{shaw_correlates_2000},
which allowed the authors to deprive individuals of rest `by gentle tapping [of their tubes] for 12 hours during the dark period'
to perform prolonged sleep deprivation\cite{shaw_stress_2002}.

\paragraph*{Mechanical}
Automatic mechanical sleep deprivation has been adopted by many laboratories, but differ in terms of
stimulus type, frequency and intensity between teams.
For instance, 
Hendricks \emph{et al.} spun fly tubes with a series for very short rotations, by short steps, 
alternating clockwise and counterclockwise. The stimulus lasted less than half a second and repeated each $T$, according to a uniform random distribution: $T \sim \mathcal{U}(30,90)~s$\cite{hendricks_need_2000}.
Shaw \emph{et al.} built a system that dropped the glass tubes ten times a minute\cite[sup. material]{shaw_correlates_2000}.
They later also developed a `Sleep Nullifying APparatus' (SNAP) that tilted animals, also every 6~s\cite{shaw_stress_2002}.
Huber \emph{et al.} implemented a similar solution with a device that would drop \glspl{dam} 1~cm (making the subjected fly fall to the bottom of its tube),
2--3 times per minute\cite{huber_sleep_2004}.	
Li \emph{et al.} rotated vials with approximately 100 animals, around their long axis, for 1~min every 2~min\cite{li_sleep_2009}.
Faville \emph{et al.} use vibration motors to deliver a train of pulses every $T \sim \mathcal{U}(20,40)~s$, with variable pulse number and duration\cite{faville_how_2015}.
In addition to devices that are built specifically to perform sleep deprivation, a number of studies have used readily available laboratory equipment such as orbital shakers and vortices\cite{kayser_sleep_2015,dubowy_genetic_2016}.

\paragraph*{Drug}
Sleep deprivation can be achieved pharmacologically, using drugs such as caffeine\cite{hendricks_modafinil_2003},
or methamphetamine\cite{andretic_dopaminergic_2005}, which has the obvious limitation that the effects of such drugs are often unspecific and may impair a broad range of neuronal and behavioural functions\cite{gilestro_pysolo_2009}.

\paragraph*{Thermogenetics} 
Thermogenetics  is a set of tools that allows neurobiologists to activate or inhibit a targeted set of neurons at a given time. 
For instance, using the UAS/Gal4 system, a population of cell expresses a thermosensitive channel.
During an experiment, when the temperature is raised, the channels open and activate targeted neurons.
This is reversed by lowering the temperature\cite{hamada_internal_2008}.
Such an approach was used to characterise wake-promoting neurons\cite{donlea_inducing_2011}.
Recently, it has also been used as a mean of sleep-depriving animals\cite{seidner_identification_2015,kayser_sleep_2015,dubowy_genetic_2016}.

\paragraph*{Social}
Instead of using a mechanical stimulus my team and myself have suggested that it could be more ecologically meaningful to deprive flies of sleep by exposing them to conspecifics.
For instance, two males kept in a confined space for several hours are effectively sleep-deprived\cite{gilestro_widespread_2009,beckwith_regulation_2017}, and males are kept restless during their interaction with a female\cite{beckwith_regulation_2017}.

\paragraph*{Light}
Light is sometimes used as a mean to keep animals awake\cite{rodrigues_short-term_2018}.
It, however, also disrupt the circadian clock, which makes it difficult to separate clock from sleep drive.

\subsubsection{Arousal probing}
The paradigms to test the arousability of animals generally involve the delivery of stimuli and the quantification of the propensity of animals to respond to them.
Often, it uses the same methods as sleep deprivation, but must also monitor their effect on flies.
Instead of delivering binary stimuli and interpreting the response\emd{}or lack thereof\emd{}as the realisation of an underlying probability,
sometimes, stimuli of ramping intensity are given until the animal wakes up\cite{hendricks_rest_2000, huber_sleep_2004, faville_how_2015}.

\paragraph*{Manual}
As described above, Hendricks and coworkers  manually delivered stimuli of increasing discrete intensity\cite{hendricks_rest_2000}.
This way, they were able to categorise sleep depth in a population of flies.
Some authors have assessed responsiveness by dropping one or two heavy objects near their experimental animals\cite{dilley_behavioral_2018}.


\paragraph*{Mechanical}
Several studies have used vibrations of increasing intensity to demonstrate that flies that had been immobile responded exclusively to the strongest disturbance\cite{shaw_correlates_2000,faville_how_2015}.
Huber \emph{et al.} designed an automatic system in which the glass tube containing each fly was hit vigorously by a flap\cite[fig.~1A]{huber_sleep_2004}.


\paragraph*{Heat}
Interestingly, escape response to gradual temperature increase was presented as a way to test sleep depth\cite[fig.~1D]{huber_sleep_2004, cirelli_reduced_2005}.
A faster variant of this approach was later used\cite{bushey_drosophila_2007}.
Surprisingly, although both studies indicate that flies do not habituate to heat\emd{}\ie{} the response probability is consistent in time\emd{}over 48~h, neither suggest using heat as an effective mean of sleep-depriving flies.

%\subsection{General features of sleep in fruit flies}

\subsection{Determinants of sleep}
\label{sec:sleep-determinants}
Like most complex phenotypes, sleep results from the interaction between genes and environmental variables.
In a broad sense, environmental variables are either external factors, such as light, temperature and social context, or
internal states such as the circadian clock, age and level of satiety.
A variety of environmental and endogenous variables have been reported to impact the amount and the structure of sleep.
In this subsection, I will describe what is known regarding the main determinants of sleep.

\subsubsection{Genetic background}

\paragraph*{Quantitative genetic} Quantitative genetic studies have described sleep amount as a very heritable trait.
Across several mutant lines the broad-sense heritability\emd{}\ie{} the proportion of the variance explained by genes\emd{}was high: $H^2 \approx 0.5$\cite{harbison_quantitative_2008}.
Similar results were reported by a genome-wide association study starting from a collection of natural variants (the \droso{} genetic reference panel)\cite{harbison_genome-wide_2013}.
Furthermore, a recent study showed a high responsiveness to artificial selection, reporting a narrow-sense heritability of $h^2 \approx 0.2$, and was able to select both for long ($\approx 650$~min) and short sleepers ($\approx 100$~min) after less than 10 generations\cite{harbison_selection_2017}.
In addition to agreeing on the description of sleep as a highly heritable phenotype, 
these three studies also recognise that sleep is a complex phenotype not only determined by multiple genes, but also
by the interaction between genotype and environment.


\paragraph*{Genetic screens}

Forward genetic\emd{}the approach that consists in first finding phenotypical variants to then discover a gene responsible for it\emd{}has 
historically been very fruitful in \droso{}.
Several landmark studies have identified short-sleeping mutants in this manner\cite{cirelli_reduced_2005,koh_identification_2008,liu_wide_2014,pfeiffenberger_cul3_2012}.
Cirelli and coworkers identified a mutation in the \emph{Shaker} gene (encoding for a potassium channel) that reduced sleep, but also suppressed sleep rebound\cite{cirelli_reduced_2005}.
Later, Koh \emph{et al.} identified \emph{sleepless}\cite{koh_identification_2008}.
Liu and colleagues identified \emph{wide awake} whose mutation reduces sleep amount. 
They presented it as an interface between the clock and the homeostat, determining the timing of sleep\cite{liu_wide_2014}.

Other single genes were discovered more serendipitously\cite{kume_dopamine_2005}, using direct approaches\cite{bushey_drosophila_2007} and reverse genetic screen\cite{pfeiffenberger_cul3_2012}.
Kume and co-workers described \emph{fumin} and the role of the neurotransmitter dopamine in the regulation of arousal\cite{kume_dopamine_2005}.
Bushey \emph{et al.} associated the previously known \emph{hyperkinetic} locus to sleep, but also to memory\cite{bushey_drosophila_2007} and
Pfeiffenberger and Allada characterised \emph{insomniac}\cite{pfeiffenberger_cul3_2012}.




%\paragraph*{Gene expression}
%53. Thimgan MS, Seugnet L, Turk J, Shaw PJ. Identification of genes associated with resilience/vulnerability to sleep deprivation and ation in Drosophila. Sleep. 2015; 38:801–14.
%54. Cirelli C, LaVaute TM, Tononi G. Sleep and wakefulness modulate gene expression in Drosophila. J Neurochem. 2005; 94:1411–9. https://doi.org/10.1111/j.1471-4159.2005.03291.x PMID: 16001966
%55. Zimmerman JE, Rizzo W, Shockley KR, Raizen DM, Naidoo N, Mackiewicz M, et al. Multiple mechanisms limit the duration of wakefulness in Drosophila brain. Physiol Genomics. 2006; 27:337–50. https://doi.org/10.1152/physiolgenomics.00030.2006 PMID: 16954408



\subsubsection{Sex}
Fruit flies are a gonochoric species where the two sexes, males and females, are determined by chromosomes.
Beyond the anatomical sexual dimorphism (\eg{} females being larger), there are noticeable differences in their physiology\emd{}for instance in metabolism, immunity\cite{hill-burns_x-linked_2009} and ageing\cite{chandegra_sexually_2017}. 
Behaviourally, both sexes are also very dimorphic, with male-specific behaviours such as male-male fighting interactions and an elaborate courtship towards females. Female-specific behaviours include their choice of egg lying site\cite{yang_drosophila_2008} and post-mating male rejection\cite{connolly_rejection_1973}.

It had been noticed by circadian researchers already that males tend to be overall less active than females.
This finding was confirmed by the early work in the sleep field. 
In particular, females are active during the day whilst males are quiescent\emd{}a period of inactivity that was wittily named the male \emph{siesta}.
This finding was corroborated by many studies, using different genetic backgrounds (\ie{} reference wild types).
A genome-wide association study has, however, showed that there is an interaction between genetics and sex, with several out-bread lines in which females sleep more than males\cite{harbison_genome-wide_2013}.


\subsubsection{Circadian clock}
Perhaps, the single most important endogenous determinant of sleep is an animal's internal clock.
In fact, the circadian field had long established that the activity of a fly cycles, even in the absence of any circadian clues (\ie{} in constant darkness and temperature).
According to the two-process model of sleep regulation (see section~\ref{sec:two-process}), the propensity of a fly to sleep is, to a large extent, modulated by its clock (process C), which acts independently from the homeostat (process S)\cite{dubowy_circadian_2017}.

These two processes can be dissociated experimentally.
For instance, mutants of the core clock genes have been shown to have unaltered sleep amount, but lose circadian regulation of sleep\cite{hendricks_gender_2003}.
Conversely, flies with very low sleep still show robust circadian activity rhythms.
The clock neurons have been shown to consolidate sleep once night has started and wake flies up in anticipation of the dawn\cite{kunst_calcitonin_2014,liu_wide_2014}.
%
%
%The nature of the interaction between clock and sleep centres is particularly studied at the neuroanatomic and molecular level. It was shown that the clock neurons drive the expression on the 
%\emph{wide wake} gene in large Ventral Lateral neurons (lLNvs), which appears to be a crucial interface
%to promote sleep immediately after the onset of the \gls{d} phase\cite{liu_wide_2014}.
%
%The mechanism by which clock neuron promote wake at the end of the night seems to involve another set of cells, the DN1 neurons\cite{kunst_calcitonin_2014}. 
%They would produce a neuropoeptide (DH31) that acts as a wake promoting factor, in the end of the night.
%However the same set of neurons are sleep promoting during the \gls{l} phase, making them a hub that integrate the clock  to control the sleep-wake balance\cite{guo_circadian_2016}.


\subsubsection{Age and development}
In many organisms, sleep amount reduces with age\cite{koh_drosophila_2006}.
When sleep was first characterised in the fruit fly, this trend was also noticed\cite{shaw_correlates_2000}.
Sleep in aged flies is not only shorter, but also more fragmented\cite{koh_drosophila_2006, zimmerman_video_2008}.
In addition, flies treated with oxidative stressors show a similar loss of sleep consolidation\cite{koh_drosophila_2006}.
Such change in sleep architecture (\ie{} number and length of bouts) is, however, not necessary and could depend on genetic and environmental factors\cite{bushey_sleep_2010, zimmerman_genetic_2012}.
Age-associated sleep fragmentation also seems related to a high-caloric diet\cite{yamazaki_high_2012}.
Young flies have been reported to sleep in a consistent location in their tube, but this preference seems to fade in with age\cite{dilley_behavioral_2018}.


In laboratory conditions, fruit flies larvae develop very rapidly, it has therefore been very difficult to maintain them long enough to study their sleep pattern.
However, a recent study has shown that larvae may engage in rest that qualifies as sleep\cite{szuperak_sleep_2018}.

\subsubsection{Diet}
Flies allocate their time to several competing needs according to their internal states. 
In some circumstances, both sleep and nutrients may be needed, and the amount and the type of food available may directly impact sleep.
Short-term starvation (12~h) has been shown to induce a large reduction of sleep amount\cite{keene_clock_2010}, possibly linked to the lipid metabolism\cite{thimgan_perilipin_2010}. 
Interestingly, sleep deprivation induced by starvation does not always result in a compensatory sleep rebound. 
In fact, it was shown that a naturally occurring allelic variant confers resistance to starvation-induced sleep loss\cite{donlea_foraging_2012}.

A complete, yeast and sucrose medium, decreased sleep in males, but increased daytime sleep in\emd{}presumably mated\emd{}females\cite{catterson_dietary_2010}, though this latter result is likely a misinterpretation resulting from the use of the single-beam \gls{dam} system\cite{garbe_changes_2016}.
It was also shown that fruit flies engaged in postprandial sleep. 
In other words, sleep is increased after a meal, and more so if it contained protein and salt\cite{murphy_postprandial_2016,murphy_simultaneous_2017}.


\subsubsection{Temperature and humidity}
Fruit flies face two important biophysical constraints: they are small and endothermic.
This implies a very limited thermal inertia and results in high poikilothermy\emd{}\ie{} their body temperature essentially equal to the temperature of their surrounding environment\cite{heinrich_insect_1995}.
In addition, their surface-area-to-mass ratio is very high, which renders them very vulnerable to the risk of desiccation\cite{prince_adaptive_1977}.
However, they are very mobile and can change their environment quickly, hence \emph{behaviourally} regulating their humidity and temperature\cite{klein_sensory_2015}.

High temperatures (shift from 25$^{\circ}$C to 30$^{\circ}$C) have interestingly been found to both increase and reduce sleep in the \gls{l} and \gls{d} phases, respectively\cite{shih_statistical_2011,ishimoto_factors_2012,parisky_reorganization_2016}.
In addition, the loss of sleep due to a night at 29$^{\circ}$C is compensated by a rebound\cite{parisky_reorganization_2016}.
These results are particularly interesting in the context of thermogenetics as the temperature will affect a wide range of physiological functions, including sleep itself. To my knowledge, there is no study on the direct effect of humidity on sleep in \droso\emd{}perhaps because it is difficult to control and measure humidity in the traditional paradigm.

\subsubsection{Light and photoperiod}
Almost all experiments on sleep in the fruit flies are carried under a 12:12~h, L:D sleep regime (with notable exceptions where flies are in DD\cite{hendricks_need_2000,joiner_sleep_2006,parisky_pdf_2008} and LL\cite{shang_light-arousal_2008}). 
However, according to the latitude, there are seasonal variations in photoperiods, that were studied in the circadian field\cite{stoleru_drosophila_2007,costa_latitudinal_1992}, but, to my knowledge, the regulation of sleep under different photo-period has not been investigated (Ko-Fan Chen, personal communication).

In addition, in the wild, there are wide spatial, daily and seasonal variations both in light intensity and quality, but their effect has not been comprehensively studied in the context of \droso{} sleep. 
One study, however, reported that flies prefer to engage in daytime sleep in shaded, darker, locations\cite{garbe_context-specific_2015}.
%\paragraph*{Effect of mating on males}

\subsubsection{Intra-specific interactions}
Besides social interactions between males and females, \droso{} has rich social interactions\cite{sokolowski_social_2010}.
For instance, males are known to fight one another for access to territory, food and mates\cite{kravitz_aggression_2015}.
In addition, emergent gregarious behaviours have been observed in populations as they spatially organise in patches\cite{navarro_pattern_1975,lefranc_non-independence_2001} 
and exhibit cooperative foraging\cite{tinette_cooperation_2004,simon_simple_2012}.

\paragraph*{Population}
A recent study reported that the structure of populations affects sleep.
Specifically, flies in a group may synchronise their sleep with one another\cite{liu_sleep_2015}.
However, this study did not have access to single animal data, making it difficult to untangle the individual from the populational effects\cite{ly_neurobiological_2018}. 
The density at which larvae were raised has also been shown to impact their sleep as adults\cite{chi_larval_2014}\emd{}though larval density is known to affect many traits such as longevity, starvation resistance and size\cite{baldal_effects_2005}.

\paragraph*{One-to-one interactions}

In a study I took part in, males were kept in a confined space for several hours either with a female or a male\cite{beckwith_regulation_2017}.
We were able to show that both conditions resulted in effective sleep deprivation for the male\cite{beckwith_regulation_2017} (see also \cite{gilestro_widespread_2009} for preliminary work).
Whilst males kept with males experienced a compensatory rebound after the removal of their competitor, 
males that had lost sleep due to their interaction (courtship and mating) with a female did not\cite{beckwith_regulation_2017}.
We were able to show that sexual clues raise males' arousal, which changes their internal state and effectively negates sleep rebound.
In a separate study, Machado \emph{et al.} characterised a wake-promoting neuronal circuit also involved in the direct regulation of courthsip\cite{lamaze_regulation_2017,stahl_sleep_2017}.

\paragraph*{Mating}

%\paragraph*{Effect of mating on females}
After mating, virgin females are known to exhibit profound physiological and behavioural switch that is induced by a peptide present in the males seminal fluid\cite{gillott_male_2003,mcgraw_genes_2004,yapici_receptor_2008}.
Namely, they produce more egg, become refractory to mating and less attractive\cite{gillott_male_2003},
their immune system is altered\cite{kapelnikov_mating_2008,innocenti_immunogenic_nodate} and 
their food preference also switched towards more salt and a high-protein diet, which they require to produce eggs\cite{ribeiro_sex_2010,walker_postmating_2015}.

Mating has been reported to affect sleep behaviour of females by several studies\cite{isaac_drosophila_2010,zimmerman_genetic_2012,garbe_context-specific_2015,garbe_changes_2016}.
Isaac \emph{et al.} scored sleep using \glspl{dam} and showed that mating reduced sleep, but only during the \gls{l} phase\emd{}night sleep being unaffected\cite{isaac_drosophila_2010}.
Garbe and co-workers also applied video tracking and multibeam and could corroborate the observation of a reduction of daytime sleep after mating\cite{garbe_context-specific_2015}. 
They also noticed a small but very consistent reduction of night sleep.
Both studies, however, used food that contained only 5\% sucrose and water,

In a follow-up article, Garbe \emph{et al.} used nutritious food (\ie{} with yeast) instead.
In this context, they found that single beam \gls{dam} failed to detect any change in daytime sleep.
They went further using a multibeam \gls{dam} and concluded that daytime sleep was indeed reduced in females after mating, but only when scored with the spatial resolution of the multibeam system.
This latter study suggests that mating induces feeding when nutritious food is available, which the single beam is likely to miss since food is on the extremity of the tube\cite{garbe_changes_2016}.
Unfortunately, the authors do not address whether night sleep, post-mating, could be altered in the presence of nutritious food.
Interestingly, the effect mating was shown to be dependant on the genetic background used\cite{zimmerman_genetic_2012}.
 % genetic background changes the effect of

\subsubsection{Inter-specific interactions}
The evolution of \droso{} has likely been driven partly by its interaction with members of other species such as predators,
pathogens, commensals and symbionts.
To the best of my knowledge, there is no work on how, for instance, the presence of predators affects sleep in the fruit fly\emd{}though it is studied
in other models\cite{lima_sleeping_2005}.
There are, however, some studies on the effect of infection, gut microbiota and symbionts.

\paragraph*{Infections and immune response}
During an infection, the behaviour of an animal may be altered for two main reasons.
Firstly, there are multiple examples of pathogens manipulating, directly on not, the behaviour of their hosts towards increasing their own fitness\cite{van_houte_walking_2013}.
Secondly, the host may face a new trade-off and alter its behaviour accordingly, for instance by modifying it preferred temperature\cite{stahlschmidt_context_2013} and diet\cite{ayres_role_2009}, which could mitigate the impact of an infection.
It has been shown that the immune system interacts with the circadian clock\cite{shirasu-hiza_interactions_2007} and sleep has been hypothesised to serve an immune function (see subsection~\ref{sec:restoration}).

There are several studies on the interactions between sleep and the immune system in \droso.
Some have suggested that infected flies reduce sleep levels\cite{mallon_immune_2014} and that sleep deprivation activates immune genes\cite{cirelli_sleep_2005}, possibly strengthening immune response and promoting survival\cite{williams_interaction_2007,kuo_acute_2014}.
In contrast, others have concluded that sleep increases immune defences.
For instances, infections have been reported to induce additional sleep\cite{kuo_sleep_2010}, which could increase survival\cite{kuo_increased_2014}.

\paragraph*{Microbiome}
The microbiota colonising \emph{Drosophila}'s gut is intimately linked to its immunity, metabolism and food preference\cite{martino_microbial_2017}.
However, it has not yet been found to play a direct or indirect role in sleep regulation.
In fact, a recent pre-print reported no effect of microbiota removal on the host's sleep\cite{selkrig_drosophila_2018}.

\paragraph*{\emph{Wolbachia}}
\emph{Wolbachia} is a genus of endosymbiotic bacteria prevalent in a wide range of arthropods\cite{zug_still_2012}.
It primarily infects their gonads and has extensively been reported to manipulate their reproduction (reviewed in\cite{werren_wolbachia_2008}) and immune system (reviewed in\cite{zug_wolbachia_2015}).
\emph{Wolbachia} can also be found in neurons and its presence has been linked to reduced aggressively in male fruit flies that host it\cite{rohrscheib_wolbachia_2015}.
Interestingly, it has recently been shown to increase arousability and decrease sleep\cite{bi_wolbachia_2018}.


\subsection{Necessity and  function of sleep in \droso{}}

The question of the role sleep to flies has been of great interest.
First and foremost, its vital nature was assessed.
Then, its global function was investigated  in the framework of the
three main hypothesis presented in subsection~\ref{sec:function}. 
Namely, energy conservation, restoration and learning. 


\subsubsection{Lethality of sleep deprivation}
\label{sec:lethality}
To my knowledge, there is only one study reporting a lethal effect of sleep deprivation in the fruit fly\cite{shaw_stress_2002}.
In this milestone article, Shaw \emph{et al.} deprived 12 wild-type flies of sleep by tapping on their tube as soon as they appeared immobile, for 70~h.
They report that 4 out of 12 animals died due to this treatment.
Despite obvious experimental and statistical limitations, the authors conclude that `sleep does indeed serve a vital biological role'\cite{shaw_stress_2002}.

The notion that sleep is vital to flies has nevertheless been supported by another line of evidence: the fact that many mutants that have reduced sleep have also shorter lifespan\cite{cirelli_is_2008}.
For instance, \emph{Shaker}\cite{cirelli_reduced_2005}, 
\emph{sleepless}\cite{koh_identification_2008},
\emph{Hyperkinetic}\cite{bushey_sleep_2011} and
\emph{insomniac}\cite{stavropoulos_insomniac_2011} mutants are all short-lived.

However, \emph{fumin} mutants, which exhibit one of the most severe short-sleeping phenotypes, have the same longevity as their genetic controls\cite{kume_dopamine_2005}.
In addition, when the expression of \emph{insomniac} is inhibited specifically in the brain of flies, they have a normal lifespan, but reduced sleep\cite{stavropoulos_insomniac_2011}.
Finally, the lifespan of artificially selected populations of low or high sleepers does not differ from the one of the control groups\cite{harbison_selection_2017}.


\subsubsection{Energy conservation and metabolism}

In an interesting study, Stahl \emph{et al.} were able to combine the traditional beam crossing assay with a device that can measure CO$_2$ production\cite{stahl_sleep-dependent_2017}. 
They could show that sleep was negatively correlated with a metabolic rate, 
and that this relationship is not only\emd{}and trivially\emd{}due to the absence of walking, 
as metabolism reduces proportionally to the time spent being immobile.
In addition, a sleep deprivation that leads to a subsequent rebound also causes in a coincident reduction of metabolism,
altogether suggesting that sleep could help flies minimise their energy expenditure.


\subsubsection{Restoration}
There is not a lot of evidence regarding of specific processes being restored by sleep in \droso{}.
As discussed above, sleep is thought to play a role in certain immune processes, and could therefore help recovery
after infection\cite{kuo_sleep_2010,kuo_increased_2014}.
Another possibility is that sleep could be involved in processes such as digestion, which may have a component of active recovery. Indeed, the fact that flies show postprandial sleep\cite{murphy_postprandial_2016} could suggest\emd{}though there are other explanations\emd{}that digestion could come at a physiological cost, possibly involving an active compensation mechanism that would be facilitated by sleep.
A last, and perhaps more compelling, example is that sleep is associated with widespread synaptic downscaling in fruit fly brain\cite{gilestro_widespread_2009}. 
Since some affected neurons are not thought to be particularly involved in learning, this process could be seen as a mechanism by which the central nervous system restores its plasticity.

\subsubsection{Learning and memory}
There is a growing literature on the involvement of sleep in learning in \droso, involving both behavioural and structural evidence.

Behaviourally, some mutants that are impaired in learning have reduced sleep\cite{liu_amnesiac_2008}.
In addition, their deficit can be rescued, to some extent, with extra sleep\cite{dissel_sleep_2015,dissel_enhanced_2017}.
Conversely, there are documented cases of low-sleepers that also have poorer learning performance\cite{bushey_drosophila_2007,seugnet_d1_2008,glou_circadian_2012}.
Furthermore, learning mutants do not seem to modulate their sleep levels according to their previous experiences\cite{ganguly-fitzgerald_waking_2006}\emd{}which wild-type animals do\cite{bushey_sleep_2011}.

Molecularly and structurally,
it has been shown that, after long periods of wakefulness, for instance due to sleep deprivation, the number of synapses (inferred by protein markers) increased, which supports the synaptic homeostasis hypothesis\cite{gilestro_widespread_2009,bushey_sleep_2011}.



\clearpage
\section{Summary, aims and scope}

\subsection{State of the art}

Sleep is a fascinating mystery that has bemused thinkers since the dawn of civilisations.
In fact, the first written tale known to date already illustrates both its importance and elusiveness.
The ancient authors, such as Aristotle and Pliny the Elder, who pioneered our comprehension of the living world, already thought that most animals, even distant ones, slept.
Today, an increasing amount of empirical evidence has since emerged to support their intuition.
Indeed, a wide range of animals has been shown to sleep, including, vertebrates, arthropods, molluscs and even cnidarians.

The conceptual proximity between sleep and death in western culture is particularity fascinating
and begs the question of whether sleep is necessary to life, an enigma that is largely unresolved. 
In fact, despite its ubiquity, the core functions of sleep themselves are still heavily debated.

\droso{} has grown an instrumental model in  our understanding of fundamental biological processes such as immunity, senescence. 
It also proved a crucial asset to understand the genetics of several 
behaviours including circadian rhythms, foraging and mating.

In the last two decades, the fruit fly emerged as a model to study sleep.
Since then, multiple genetic and environmental determinants have been characterised,
altogether leading to the notion that sleep is a dynamic process that depends on contexts such as
sex, mating status, social clue, interaction with pathogens, temperature and diet.
It has been suggested that sleep helps conserving energy, restore the immune and nervous systems, and promotes learning in flies. 
However, the question of whether fly can live without sleep remains largely open\emd{}considering the lack of conclusive evidence.


The field of \droso{} research benefits from a vast and constantly evolving palette of scientific tools such as genetic expression systems, themogenetics, confocal imaging, genomics, transcriptomics and many others.
In comparison, the paradigms used to score and alter sleep lag behind.
Indeed, despite recent advances in hardware and software for video tracking, which can 
detect movement, immobility\emd{}and therefore sleep\emd{}has been almost exclusively inferred with the \gls{dam}, a device designed to detect large-amplitude walking activity, which cannot account for low-amplitude movements such as feeding and grooming.
Likewise, the sleep deprivation paradigms employed to understand homeostasis have traditionally been unspecific as they consist of untargeted, frequent, mechanical disturbance.

\subsection{The scope of this thesis}

In the next chapter of the thesis herein, I will present the ethoscope platform, a practical solution I developed with my collaborators to address some of the methodological limitations associated with the detection of movement and ultimately sleep. I designed it aiming to deliver a scalable, open and modular tool for the community, not only to score sleep in \droso{} but, in principle, to analyse the behaviour of small animals in general.


The development of the ethoscope ultimately resulted in the acquisition of
large individual datasets.
I wanted to enrich the traditional analysis of movement with a modern approach based on data sciences, which made me realise the absence of a general programmatic framework
to, specifically, process and visualise high-throughput behavioural data from several inputs. 
I, therefore, led the development of \texttt{rethomics} as a collection of \texttt{R} packages aiming to fill this niche. 
In the third chapter, I will describe its scope and originality.


I was very interested in understanding how the use of my novel tracking tool could yield informative variables and ultimately affect the quality of our biological conclusions.
I will present, in my fourth chapter, a series of experiments going in this direction.
I specifically focussed on acquiring large amounts of data for wild-type animals, comparing them between \gls{dam} and ethoscopes,
enriching the behavioural space, characterising the consistency of behaviour over time and using positional data in addition to discrete behavioural states.

In my fifth chapter, I will show the results I obtained when employing the ethoscope's modularity to perform dynamic (\ie{} real-time) sleep deprivation.
Such a technique enabled me to prevent animals from being quiescent by startling them only when they were immobile. 
This approach being more parsimonious, 
it eventually allowed for a final experiment in which 
flies were chronically sleep-deprived, ultimately addressing the crucial question of the vital necessity of sleep.
